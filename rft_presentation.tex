% beamer class with 16:9 aspect ratio
\documentclass[aspectratio=169]{beamer}

% load all packages
% input encoding
\usepackage[utf8]{inputenc}

%--------------------------------------------
% Font encoding
%--------------------------------------------
\usepackage[T1]{fontenc} % Schriftherkunft

%--------------------------------------------
% Code listings
%--------------------------------------------
\usepackage{listings}
\usepackage{xcolor}

% For better code display with tcolorbox and minted
\usepackage{tcolorbox}
\tcbuselibrary{minted,breakable,xparse,skins}
\usepackage{minted}

% Define colors for code boxes
\definecolor{bg}{gray}{0.95}

% Define beautiful code box - optimized for beamer slides
\DeclareTCBListing{mintedbox}{O{}m!O{}}{%
  breakable=true,
  listing engine=minted,
  listing only,
  minted language=#2,
  minted style=default,
  minted options={%
    linenos,
    gobble=0,
    breaklines=true,
    breakafter=,,
    fontsize=\footnotesize,
    numbersep=5pt,
    autogobble=true,
    #1},
  boxsep=0pt,
  left skip=0pt,
  right skip=0pt,
  left=20pt,
  right=0pt,
  top=2pt,
  bottom=2pt,
  arc=3pt,
  leftrule=0pt,
  rightrule=0pt,
  bottomrule=1pt,
  toprule=1pt,
  colback=bg,
  colframe=orange!70,
  width=\textwidth-10pt,
  nobeforeafter=true,
  size=minimal,
  enhanced,
  overlay={%
    \begin{tcbclipinterior}
    \fill[orange!20!white] (frame.south west) rectangle ([xshift=15pt]frame.north west);
    \end{tcbclipinterior}},
  #3,
}

% Basic listings settings (for compatibility)
\lstset{
  basicstyle=\ttfamily\small,
  keywordstyle=\color{blue},
  commentstyle=\color{gray},
  breaklines=true,
  showstringspaces=false,
  frame=single,
  frameround=tttt
}

% package for the English language
\usepackage[english]{babel}

% default graphics path
\graphicspath{{logos/}{figures/}}

% package to create diagrams and plot with TikZ
\usepackage{pgfplots}
% always use the newest version
\pgfplotsset{compat=newest}
% define the style of the connection of two lines
\pgfplotsset{every axis/.append style={line join=bevel}}
% generate a new style template
\mode<beamer>{
	\pgfplotsset{
		beamer/.style={
			width=0.8\textwidth,
			height=0.45\textwidth,
			legend style={font=\scriptsize},
			tick label style={font=\footnotesize},
			label style={font=\small},
			max space between ticks=28,
		}
	}
}
\mode<handout>{
	\pgfplotsset{
		beamer/.style={
			width=0.8\textwidth,
			height=0.45\textwidth,
			legend style={font=\scriptsize},
			tick label style={font=\footnotesize},
			label style={font=\small},
			max space between ticks=25,
		}
	}
}
\mode<article>{
	\pgfplotsset{
		beamer/.style={
			width=0.8\textwidth,
			height=0.45\textwidth,
			max space between ticks=35,
		}
	}
}
% format and size template for two plots side-by-side
\pgfplotsset{
	scriptsize/.style={
		width=0.34\textwidth,
		height=0.1768\textwidth,
		legend style={font=\scriptsize},
		tick label style={font=\scriptsize},
		label style={font=\footnotesize},
		title style={font=\footnotesize},
		every axis title shift=0pt,
		max space between ticks=25,
		every mark/.append style={mark size=7},
		major tick length=0.1cm,
		minor tick length=0.066cm,
	}
}
\pgfplotsset{
	small/.style={
		width=6.5cm,
		height=,
		tick label style={font=\footnotesize},
		label style={font=\small},
		legend style={font=\footnotesize},
		max space between ticks=30,
	}
}
% align legends entries to the left
\pgfplotsset{legend cell align=left}
% draw a main grid
\pgfplotsset{xmajorgrids}
\pgfplotsset{ymajorgrids}
% number of minor ticks between two major tickmarks
%\pgfplotsset{minor x tick num={3}}
%\pgfplotsset{minor y tick num={3}}
% draw a minor grid
%\pgfplotsset{xminorgrids}
%\pgfplotsset{yminorgrids}
% scale only the axis using a certain width and height
\pgfplotsset{scale only axis}
% define colors like in MATLAB
\definecolor{matlab1}{rgb}{0,0,1}
\definecolor{matlab2}{rgb}{0,0.5,0}
\definecolor{matlab3}{rgb}{1,0,0}
\definecolor{matlab4}{rgb}{0,0.75,0.75}
\definecolor{matlab5}{rgb}{0.75,0,0.75}
\definecolor{matlab6}{rgb}{0.75,0.75,0}
\definecolor{matlab7}{rgb}{0.25,0.25,0.25}
% define a color cycle list like in MATLAB
\pgfplotscreateplotcyclelist{matlab}{
	{matlab1,solid},
	{matlab2,dashed},
	{matlab3,dashdotted},
	{matlab4,dotted},
	{matlab5,densely dashed},
	{matlab6,densely dashdotted},
	{matlab7,densely dotted}%this prevents an error
}
% use the earlier defined color list
\pgfplotsset{cycle list name=matlab}
% use the standard color list of pgfplots
%\pgfplotsset{cycle list name=color list}
% use only grayscale lines
%\pgfplotsset{cycle list name=linestyles}
% use a line width of 1pt
\pgfplotsset{every axis plot/.append style={line width=1pt}}
% use a comma as the dot separator for german documents
\addto\extrasngerman{\pgfplotsset{/pgf/number format/.cd,set decimal separator={{{,}}}}}
% use a half space as the 1000 separator
\pgfplotsset{/pgf/number format/.cd,1000 sep={\,}}
% define legend positions
\pgfplotsset{/pgfplots/legend pos/north/.style={/pgfplots/legend style={at={(0.50,0.97)},anchor=north}}}
\pgfplotsset{/pgfplots/legend pos/south/.style={/pgfplots/legend style={at={(0.50,0.03)},anchor=south}}}
\pgfplotsset{/pgfplots/legend pos/east/.style={/pgfplots/legend style={at={(0.97,0.50)},anchor=east}}}
\pgfplotsset{/pgfplots/legend pos/west/.style={/pgfplots/legend style={at={(0.03,0.50)},anchor=west}}}
\pgfplotsset{/pgfplots/legend pos/outer north/.style={/pgfplots/legend style={at={(0.50,1.03)},anchor=south}}}

% package for SI units
\usepackage{siunitx}
% real fractions
\sisetup{per-mode=fraction,mode=math}
% decimal marker in dependence from the language
\addto\extrasngerman{\sisetup{output-decimal-marker={,}}}
\addto\extrasenglish{\sisetup{output-decimal-marker={.}}}
% range phrase in dependence from the language
\addto\extrasngerman{\sisetup{range-phrase={ bis~}}} 
\addto\extrasenglish{\sisetup{range-phrase={ to~}}}

% A stand-alone package that implements several commands for including external animation and sound
% files in a pdf document. The package can be used together with both dvips plus ps2pdf and pdflatex,
% though the special sound support is available only in pdflatex.
\usepackage{multimedia}

% package to dissable floating in the article mode
\usepackage{float}

% package to change the line spacing
\usepackage{setspace}

% package for nice fractions in the text mode
\usepackage{xfrac}
% Comment out the problematic command
% \UseCollection{xfrac}{plainmath}

% nicer tables
\usepackage{booktabs}

% nicer quotation marks
\usepackage{csquotes}

% define labeling environment
\makeatletter
\newenvironment{labeling}[2][]{%
  \def\sc@septext{#1}%
  \list{}{\settowidth{\labelwidth}{{%
        #2%
          \sc@septext%
      }}%
    \leftmargin\labelwidth \advance\leftmargin by \labelsep
    \let\makelabel\labelinglabel
  }%
}{%
  \endlist
}
\newcommand\labelinglabel[1]{%
  #1\hfil
    \sc@septext%
}
\makeatother

% 暂时注释掉BibLaTeX设置,因为没有参考文献需要引用
% \usepackage[%
% 	style=ieee,%				format as used at the IEEE
% 	backend=biber,%			suppress warning
% ]{biblatex}

% database of references
% \addbibresource{literature.bib}

% How to remove some pages from navigation bullets in beamer?
% see: https://tex.stackexchange.com/questions/37127/how-to-remove-some-pages-from-the-navigation-bullets-in-beamer
\makeatletter
\let\beamer@writeslidentry@miniframeson=\beamer@writeslidentry
\def\beamer@writeslidentry@miniframesoff{%
	\expandafter\beamer@ifempty\expandafter{\beamer@framestartpage}{}% does not happen normally
	{%else
		% removed \addtocontents commands
		\clearpage\beamer@notesactions%
	}
}
\newcommand*{\miniframeson}{\let\beamer@writeslidentry=\beamer@writeslidentry@miniframeson}
\newcommand*{\miniframesoff}{\let\beamer@writeslidentry=\beamer@writeslidentry@miniframesoff}
\makeatother

% Distance between points of subsections in miniframe navigation
% siehe: http://tex.stackexchange.com/questions/288548/beamer-miniframes-horizontal-space-between-dots-when-using-the-compress-option
\makeatletter
\def\slideentry#1#2#3#4#5#6{%
  %section number, subsection number, slide number, first/last frame, page number, part number
  \ifnum#6=\c@part\ifnum#2>0\ifnum#3>0%
    \ifbeamer@compress%
      \advance\beamer@xpos by1\relax%
      \ifnum1=#3                        %%% NOTICE
				\ifnum1=#2\else
					\advance\beamer@xpos by1\relax% %%% THESE
				\fi
      \fi                               %%% LINE
    \else%
      \beamer@xpos=#3\relax%
      \beamer@ypos=#2\relax%
    \fi%
  \hbox to 0pt{%
    \beamer@tempdim=-\beamer@vboxoffset%
    \advance\beamer@tempdim by-\beamer@boxsize%
    \multiply\beamer@tempdim by\beamer@ypos%
    \advance\beamer@tempdim by -.05cm%
    \raise\beamer@tempdim\hbox{%
      \beamer@tempdim=\beamer@boxsize%
      \multiply\beamer@tempdim by\beamer@xpos%
      \advance\beamer@tempdim by -\beamer@boxsize%
      \advance\beamer@tempdim by 1pt%
      \kern\beamer@tempdim
      \global\beamer@section@min@dim\beamer@tempdim
      \hbox{\beamer@link(#4){%
          \usebeamerfont{mini frame}%
          \ifnum\c@section=#1%
            \ifnum\c@subsection=#2%
              \usebeamercolor[fg]{mini frame}%
              \ifnum\c@subsectionslide=#3%
                \usebeamertemplate{mini frame}%\beamer@minislidehilight%
              \else%
                \usebeamertemplate{mini frame in current subsection}%\beamer@minisliderowhilight%
              \fi%
            \else%
              \usebeamercolor{mini frame}%
              %\color{fg!50!bg}%
              \usebeamertemplate{mini frame in other subsection}%\beamer@minislide%
            \fi%
          \else%
            \usebeamercolor{mini frame}%
            %\color{fg!50!bg}%
            \usebeamertemplate{mini frame in other subsection}%\beamer@minislide%
          \fi%
        }}}\hskip-10cm plus 1fil%
  }\fi\fi%
  \else%
  \fakeslideentry{#1}{#2}{#3}{#4}{#5}{#6}%
  \fi\ignorespaces
}


% Use the modified OVGU theme with Zhijiang Lab logo
\usepackage{beamer_ovgu_169}

% 禁用参考文献功能,避免bbl文件缺失警告
\providecommand{\bibfont}{}
\renewcommand{\bibfont}{\normalfont\footnotesize}
\AtEndDocument{\def\bibfont{}}
\let\printbibliography\relax

\title[RFT Guide 2025]{The Complete Guide to Reinforcement Fine-Tuning}
\author{Chen Xingqiang}
\institute[Yizhu Intelligent Ltd]{
	Yizhu Intelligent Ltd\\
	Hangzhou, China
}
\date[2025-04-25]{April 25, 2025}

\begin{document}

\begin{frame}
	\maketitle
\end{frame}

\begin{frame}[label=inhalt]{Overview}
	\tableofcontents
\end{frame}

\section{Reinventing AI with DeepSeek-R1 and Reinforcement Learning}

\begin{frame}
	\frametitle{Introduction to Reinforcement Learning}
	\begin{block}{The Self-Improvement Paradigm}
		\begin{itemize}
			\item Reinforcement learning introduces a feedback-driven mechanism for training AI
			\item Rather than relying on labeled examples, RL agents learn by:
			\begin{itemize}
				\item \textbf{Exploration}: The model attempts multiple strategies or actions
				\item \textbf{Reward}: Each action yields reward signals that guide future choices
			\end{itemize}
			\item More closely aligns with how humans naturally learn through trial, error, and feedback
		\end{itemize}
	\end{block}
\end{frame}

\begin{frame}
	\frametitle{Introduction to Reinforcement Learning - Continued}
	\begin{block}{The Self-Improvement Paradigm (cont.)}
		\begin{itemize}
			\item Opens the door to continual learning and deeper reasoning capabilities
			\item AI is no longer a static endeavor—models can evolve and adapt after deployment
		\end{itemize}
	\end{block}
	\begin{block}{From Static to Dynamic Learning}
		\begin{itemize}
			\item Traditional approaches rely on enormous labeled datasets for memorization
			\item RL shifts focus to learning strategies and reasoning patterns
			\item The "environment" can be any context where feedback signals guide improvement
		\end{itemize}
	\end{block}
\end{frame}

\begin{frame}
	\frametitle{The Significance of DeepSeek-R1}
	\begin{block}{DeepSeek-R1: A Model Designed to Reason}
		\begin{itemize}
			\item \textbf{Adaptive Reward Structures}: Multiple reward functions focusing on accuracy, efficiency, and creativity
			\item \textbf{Iterative Refinement}: Reward-based feedback loops emphasizing what works best in practice
			\item \textbf{Breaking Performance Barriers}: Continuous learning allows it to outperform traditional LLMs
			\item \textbf{Open-Source Advantage}: Unlike OpenAI's closed-source O1, DeepSeek-R1 shares weights and training approaches
		\end{itemize}
	\end{block}
\end{frame}

\begin{frame}
	\frametitle{DeepSeek-R1: Technical Implementation}
	\begin{block}{Active Exploration Mechanisms}
		\begin{itemize}
			\item Instead of passively ingesting training data, performs active exploration
			\item Training involves not just error backpropagation but reward-based feedback loops
			\item \textbf{Exploration Strategy}: Balances trying new approaches with exploiting proven methods
			\item \textbf{Multi-Objective Optimization}: Learns to balance accuracy, efficiency, and creativity
		\end{itemize}
	\end{block}
	\begin{block}{Democratizing Reasoning Models}
		\begin{itemize}
			\item Open-sourcing enables engineering teams to customize their own reasoning models
			\item Released weights allow deeper inspection and understanding of model behavior
			\item Community contributions accelerate innovation in reasoning capabilities
		\end{itemize}
	\end{block}
\end{frame}

\begin{frame}
	\frametitle{DeepSeek-R1 vs. Traditional Models}
	\begin{block}{Key Differences:}
		\begin{itemize}
			\item Traditional models train once on static datasets and freeze parameters
			\item DeepSeek-R1 takes a dynamic approach to learning
		\end{itemize}
	\end{block}
\end{frame}

\begin{frame}
	\frametitle{DeepSeek-R1 vs. Traditional Models (cont.)}
	\begin{block}{DeepSeek-R1 Advantages:}
		\begin{itemize}
			\item Learns dynamically, responding to changes in requirements
			\item Requires fewer labeled data points by learning from rewards
			\item Minimizes the risk of stagnation through continuous learning
		\end{itemize}
	\end{block}
	\begin{block}{Key Insight:}
		\begin{itemize}
			\item "One-and-done" training becoming a relic of the past
		\end{itemize}
	\end{block}
\end{frame}

\section{Reinforcement Fine-Tuning vs. Supervised Fine-Tuning}

\begin{frame}
	\frametitle{What is Reinforcement Fine-Tuning (RFT)?}
	\begin{block}{Combining Fine-Tuning with RL}
		\begin{itemize}
			\item RFT combines strengths of pre-trained LLMs with feedback-driven RL
			\item Core process:
			\begin{enumerate}
				\item Start with a pre-trained model with general knowledge
				\item Define a reward function for target metrics
				\item Iteratively fine-tune using RL techniques
			\end{enumerate}
			\item Enables customization of open-source LLMs with minimal data
			\item Turns general models into powerful reasoning models for specific tasks
		\end{itemize}
	\end{block}
\end{frame}

\begin{frame}
	\frametitle{RFT vs. SFT Comparison}
	\begin{table}
		\centering
		\begin{tabular}{|l|l|l|}
			\hline
			\textbf{Factor} & \textbf{RFT} & \textbf{SFT} \\
			\hline
			Data Requirements & Minimal labeled data & Needs 1,000+ rows \\
			\hline
			Adaptability & Continuous improvement & Limited by labeled data \\
			\hline
			Exploration & Actively tries new strategies & Relies on fixed examples \\
			\hline
			Performance & Continual progress & Reaches plateau \\
			\hline
			Error Handling & Learns from mistakes & Repeats errors in data \\
			\hline
			Training Complexity & Higher (needs reward function) & Lower (just needs examples) \\
			\hline
		\end{tabular}
		\caption{Comparison of RFT and SFT approaches}
	\end{table}
\end{frame}

\begin{frame}
	\frametitle{When RFT Wins}
	\begin{block}{RFT Excels with Scarce Data}
		\begin{itemize}
			\item Removes need for labeled data, relies on objective correctness
			\item Outperforms SFT with small datasets (dozens of examples)
			\item Resists overfitting by learning robust strategies
			\item Best use cases:
			\begin{itemize}
				\item Code Transpilation (e.g., Java to Python)
				\item Game Strategy (Chess, Wordle)
				\item Medical Diagnosis (learning from feedback at decision points)
			\end{itemize}
		\end{itemize}
	\end{block}
\end{frame}

\begin{frame}
	\frametitle{Quantitative Performance Comparison}
	\begin{block}{RFT vs SFT by Dataset Size}
		\begin{itemize}
			\item \textbf{10 Examples}: RFT improved base model by 18\%, SFT showed minimal gains
			\item \textbf{50 Examples}: RFT showed 42\% improvement over baseline
			\item \textbf{100 Examples}: RFT improvement jumped to 60\%, 3x better than SFT
			\item The smaller the dataset, the greater RFT's advantage over SFT
		\end{itemize}
	\end{block}
	\begin{figure}[ht]
		\centering
		\begin{tikzpicture}[scale=0.5, transform shape]
			\begin{axis}[
				xlabel={Dataset Size},
				ylabel={Performance Improvement (\%)},
				legend style={at={(0.5,-0.15)}, anchor=north},
				ymin=0,
				ymax=70,
				grid=both,
				minor tick num=1,
				width=8cm,
				height=5cm,
				legend columns=2,
				scaled ticks=false,
				font=\footnotesize,
				xtick={10, 50, 100}]
				\addplot[color=blue, mark=*,line width=1.2pt] coordinates {
					(10,18) (50,42) (100,60)
				};
				\addplot[color=red, mark=square*,line width=1.2pt] coordinates {
					(10,5) (50,15) (100,20)
				};
				\legend{RFT, SFT};
			\end{axis}
		\end{tikzpicture}
		\caption{Performance Improvement vs. Dataset Size}
	\end{figure}
\end{frame}

\begin{frame}
	\frametitle{When to Use RFT vs. SFT}
	\begin{block}{Decision Factors:}
		\begin{itemize}
			\item \textbf{Data Availability}: RFT for limited data; SFT for abundant labeled data
			\item \textbf{Task Complexity}: RFT for tasks with clear success criteria
			\item \textbf{Performance Goals}: RFT for continuous improvement; SFT for stable results
			\item \textbf{Verifiability}: RFT excels when outcomes can be objectively measured
			\item \textbf{Resource Constraints}: SFT simpler to implement initially but requires more data
		\end{itemize}
	\end{block}
	\begin{block}{Real-World Applications:}
		\begin{itemize}
			\item \textbf{Coding Assistants}: RFT trains models to write code that compiles and passes tests
			\item \textbf{Data Analysis}: RFT improves query generation that produces correct results
			\item \textbf{Reasoning Tasks}: RFT enhances step-by-step problem-solving capabilities
		\end{itemize}
	\end{block}
\end{frame}

\section{Accelerating Reasoning Models with Turbo LoRA}

\begin{frame}
	\frametitle{The Challenge with Reasoning Models}
	\begin{block}{Throughput Issues:}
		\begin{itemize}
			\item Reasoning models push AI's problem-solving capabilities
			\item But advanced reasoning comes at a cost: slow throughput
			\item Reasoning models "think" by generating many tokens
			\item Multiple intermediate computations increase processing time
			\item Production deployment requires addressing these challenges
		\end{itemize}
	\end{block}
	\begin{block}{Real-World Performance Bottlenecks}
		\begin{itemize}
			\item Reasoning models can be 2-3x slower than comparable non-reasoning models
			\item Higher latency significantly impacts user experience and cost-efficiency
			\item Traditional acceleration methods often compromise reasoning quality
			\item Need specialized solutions that preserve reasoning ability while boosting speed
		\end{itemize}
	\end{block}
\end{frame}

\begin{frame}
	\frametitle{LoRA and Turbo LoRA}
	\begin{block}{LoRA: Low-Rank Adaptation}
		\begin{itemize}
			\item Fine-tunes large models with small set of trainable parameters
			\item Preserves original weights, maintaining pre-trained knowledge
			\item Dramatically reduces memory requirements for fine-tuning
			\item Enables efficient adaptation of models for specialized tasks
		\end{itemize}
	\end{block}
	\begin{block}{Turbo LoRA: 2-4x Faster Reasoning}
		\begin{itemize}
			\item Uses speculative decoding along with proprietary optimizations
			\item Predicts multiple tokens in parallel, then verifies them
			\item Maintains output quality while generating text faster
			\item Result: Multiple tokens per step instead of one at a time
		\end{itemize}
	\end{block}
\end{frame}

\begin{frame}
	\frametitle{How Turbo LoRA Works}
	\begin{columns}
		\begin{column}{0.48\textwidth}
			\begin{block}{Technical Implementation}
				\begin{enumerate}
					\item Small, fast "speculator" model predicts several tokens in parallel
					\item Main model verifies predicted tokens
					\item Correct tokens are instantly used
					\item Only incorrect tokens need recalculation
				\end{enumerate}
			\end{block}
		\end{column}
		\begin{column}{0.48\textwidth}
			\begin{block}{Practical Benefits}
				\begin{itemize}
					\item Zero difference in final output quality
					\item Applied to DeepSeek-R1-distill-qwen-32b model
					\item Demonstrated 2-3x throughput improvement
					\item Can be applied to any reasoning model
				\end{itemize}
			\end{block}
		\end{column}
	\end{columns}
	\vspace{0.5cm}
	\begin{center}
		\begin{tikzpicture}
			\draw[thick,->] (0,0) -- (9,0) node[right] {Time};
			% Standard generation
			\draw[thick] (0,1) -- (9,1);
			\foreach \x in {1,2,3,4,5,6,7,8}
				\draw[fill=gray!30] (\x,0.7) rectangle (\x+0.5,1.3) node[pos=.5] {$t_\x$};
			% Turbo generation
			\draw[thick] (0,2.5) -- (9,2.5);
			\draw[fill=blue!20] (1,2.2) rectangle (4,2.8) node[pos=.5] {$t_{1-4}$};
			\draw[fill=blue!20] (4.5,2.2) rectangle (7.5,2.8) node[pos=.5] {$t_{5-8}$};
			% Labels
			\node at (-1,1) {Standard:};
			\node at (-1,2.5) {Turbo:};
		\end{tikzpicture}
	\end{center}
\end{frame}

\begin{frame}
	\frametitle{Benefits of Turbo for Large Reasoning Models}
	\begin{block}{Making Real-Time AI Feasible}
		\begin{itemize}
			\item 2-3x speedup makes reasoning models viable for:
			\begin{itemize}
				\item AI-powered customer support
			\end{itemize}
		\end{itemize}
	\end{block}
\end{frame}

\begin{frame}
	\frametitle{Benefits of Turbo for Large Reasoning Models (cont.)}
	\begin{block}{Making Real-Time AI Feasible (cont.)}
		\begin{itemize}
			\item 2-3x speedup makes reasoning models viable for:
			\begin{itemize}
				\item AI copilots for developers
				\item Healthcare AI assistants
			\end{itemize}
			\item Lower GPU costs: fewer GPUs needed for the same workload
		\end{itemize}
	\end{block}
\end{frame}

\begin{frame}
	\frametitle{Benefits of Turbo for Large Reasoning Models (cont.)}
	\begin{block}{Real-World Impact}
		\begin{itemize}
			\item Makes inference of reasoning models feasible for almost any organization
			\item Typical inference setup: 2-3x reduction in required GPU capacity
			\item Cost savings scale with deployment size - larger deployments save more
		\end{itemize}
	\end{block}
	\begin{block}{Implementation}
		\begin{itemize}
			\item Enables use of smaller, more efficient models without sacrificing capability
			\item For detailed implementation tutorial: \texttt{https://predibase.com/blog/turbo-lora}
		\end{itemize}
	\end{block}
\end{frame}

\section{Tutorial: Using RFT to Write CUDA Kernels}

\begin{frame}
	\frametitle{Why GPU Code Generation is Hard}
	\begin{block}{Challenges of GPU Coding:}
		\begin{itemize}
			\item \textbf{Parallel Architecture}: Harnessing hundreds/thousands of cores
			\item \textbf{Memory Hierarchy}: Balancing register, shared, and global memory
			\item \textbf{Thread Synchronization}: Avoiding race conditions and deadlocks
			\item Minor errors can degrade performance or crash the GPU
			\item Teaching AI best practices limited by scarcity of examples
		\end{itemize}
	\end{block}
	\begin{block}{Why Traditional Approaches Fall Short}
		\begin{itemize}
			\item Limited high-quality examples of CUDA/Triton kernels available
			\item Supervised learning requires thousands of code pairs for effective learning
			\item GPU code is highly technical with many edge cases and optimizations
			\item Subtle syntax and semantic errors can cause compilation failures
		\end{itemize}
	\end{block}
\end{frame}

\begin{frame}
	\frametitle{Why RFT is Well-Suited for Code Generation}
	\begin{block}{Perfect Fit for RL:}
		\begin{itemize}
			\item \textbf{No Large Labeled Dataset Needed}: Started with just handful of examples
			\item \textbf{Code is Verifiable}: Can deterministically test compilation and correctness
			\item \textbf{Large Search Space}: RL balances exploring new implementations with exploiting proven approaches
			\item Started with just 13 hand-curated examples of PyTorch functions
		\end{itemize}
	\end{block}
\end{frame}

\begin{frame}
	\frametitle{Setting Up the CUDA Task: Minimal Dataset}
	\begin{block}{Dataset Composition}
		\begin{itemize}
			\item Each example contained:
			\begin{itemize}
				\item A PyTorch function (e.g., matrix multiply or activation function)
				\item A set of test cases to verify correctness
			\end{itemize}
			\item Example functions included matrix operations, activations, and element-wise operations
			\item Test cases covered edge cases, different sizes, and boundary conditions
		\end{itemize}
	\end{block}
	\begin{block}{PyTorch to Triton Example}
	\begin{itemize}
		\item \textbf{PyTorch function:}
		\begin{quote}
		\texttt{def add(x, y): return x + y}
		\end{quote}
	\end{itemize}
	\end{block}
\end{frame}

\begin{frame}
	\frametitle{Setting Up the CUDA Task: Triton Implementation}
	\begin{block}{Target Triton Kernel}
	\begin{quote}
	\texttt{@triton.jit}\\  
	\texttt{def add\_kernel(x\_ptr, y\_ptr, output\_ptr, n\_elements):}\\  
	\texttt{\phantom{xxxx}pid = tl.program\_id(0)}\\  
	\texttt{\phantom{xxxx}block\_size = 128}\\  
	\texttt{\phantom{xxxx}offsets = pid * block\_size + tl.arange(0, block\_size)}
	\end{quote}
	\end{block}
\end{frame}

\begin{frame}
	\frametitle{Setting Up the CUDA Task: Triton Implementation (cont.)}
	\begin{block}{Target Triton Kernel (cont.)}
	\begin{quote}
	\texttt{\phantom{xxxx}mask = offsets < n\_elements}\\  
	\texttt{\phantom{xxxx}x = tl.load(x\_ptr + offsets, mask=mask)}\\  
	\texttt{\phantom{xxxx}y = tl.load(y\_ptr + offsets, mask=mask)}\\  
	\texttt{\phantom{xxxx}output = x + y}\\  
	\texttt{\phantom{xxxx}tl.store(output\_ptr + offsets, output, mask=mask)}
	\end{quote}
	\end{block}
\end{frame}

\begin{frame}
	\frametitle{Defining Rewards for Code Generation}
	\begin{block}{Multi-Level Reward Structure (1)}
		\begin{description}
			\item[\textbf{Reward 1:}] \textbf{Formatting (0.1-0.3)}
		\end{description}
		\begin{itemize}
			\item Code structure, imports, tags
			\item Partial credit for good Triton semantics
			\item Reasonable variable names and code organization
		\end{itemize}
	\end{block}
	\begin{block}{Multi-Level Reward Structure (2)}
		\begin{description}
			\item[\textbf{Reward 2:}] \textbf{Compilation (0.3-0.6)}
		\end{description}
		\begin{itemize}
			\item Code that executes without errors
			\item No runtime exceptions or syntax errors
			\item Properly imports required dependencies
		\end{itemize}
	\end{block}
\end{frame}

\begin{frame}
	\frametitle{Defining Rewards for Code Generation (cont.)}
	\begin{block}{Multi-Level Reward Structure (3)}
		\begin{description}
			\item[\textbf{Reward 3:}] \textbf{Correctness (0.6-1.0)}
		\end{description}
	\end{block}
	\begin{itemize}
		\item Output matches PyTorch function on test inputs
		\item Anti-reward-hacking measures (checking for hardcoded outputs)
		\item Proper handling of edge cases and different input shapes
	\end{itemize}
	\vspace{0.5em}
	\begin{block}{Benefits of This Reward Structure}
		\begin{itemize}
			\item Provides clear progression path for the model
			\item Allows partial credit for partial solutions
			\item Mirrors how humans learn to code
		\end{itemize}
	\end{block}
\end{frame}

\begin{frame}
	\frametitle{Example Implementation of Reward Function}
	\begin{block}{Reward Function Logic - Part 1}
		\begin{enumerate}
			\item \textbf{Formatting check:}
			\begin{itemize}
				\item If code has proper Triton syntax $\rightarrow$ reward = 0.2
			\end{itemize}
			\item \textbf{Compilation check:}
			\begin{itemize}
				\item If code compiles successfully $\rightarrow$ reward = 0.5
				\item If compilation fails $\rightarrow$ return current reward
			\end{itemize}
		\end{enumerate}
	\end{block}
\end{frame}

\begin{frame}
	\frametitle{Example Implementation of Reward Function (cont.)}
	\begin{columns}
		\begin{column}{0.48\textwidth}
			\begin{block}{Reward Function Logic - Part 2}
				\begin{enumerate}\setcounter{enumi}{2}
					\item \textbf{Correctness check:}
					\begin{itemize}
						\item For each test case, compare PyTorch and Triton results
						\item If outputs match $\rightarrow$ reward = 1.0
						\item Otherwise $\rightarrow$ reward = 0.6
					\end{itemize}
				\end{enumerate}
			\end{block}
		\end{column}
		\begin{column}{0.48\textwidth}
			\begin{block}{Key Elements}
				\begin{itemize}
					\item Multi-stage evaluation process
					\item Each stage builds on the previous
					\item Partial rewards for partial successes
					\item Deterministic verification through test cases
					\item Numerically scaled from 0.0 to 1.0
					\item Multiple test cases ensure robustness
				\end{itemize}
			\end{block}
		\end{column}
	\end{columns}
\end{frame}

\begin{frame}
	\frametitle{Training Loop and Results}
	\begin{block}{How GRPO (Gradient-based Reward Policy Optimization) Works:}
		\begin{itemize}
			\item \textbf{Generate}: Multiple completions per prompt using sampling
			\item \textbf{Evaluate}: Run reward checks on each completion
			\item \textbf{Update}: Compute advantages and backpropagate signals
			\item \textbf{Repeat}: Model refines strategy to maximize rewards
		\end{itemize}
	\end{block}
	\begin{block}{Results:}
		\begin{itemize}
			\item 53\% accuracy on held-out examples after 5,000 steps
			\item 3x higher correctness rate than OpenAI o1 and DeepSeek-R1
			\item 4x better performance than Claude 3.7 Sonnet
		\end{itemize}
	\end{block}
\end{frame}

\begin{frame}
	\frametitle{Performance Comparison with Leading Models}
	\begin{columns}
		\begin{column}{0.48\textwidth}
			\begin{center}
				\begin{tikzpicture}[scale=0.75, transform shape]
					\begin{axis}[
						ybar,
						axis lines=middle,
						bar width=15pt,
						ylabel=Accuracy (\%),
						symbolic x coords={RFT, o1, DeepSeek-R1, Claude 3.7},
						xtick=data,
						ytick={0,10,20,30,40,50,60},
						ymin=0,
						ymax=60,
						nodes near coords,
						width=7cm,
						height=5cm,
						legend style={at={(0.5,-0.15)}, anchor=north},
						]
						\addplot[fill=blue!70] coordinates {(RFT, 53) (o1, 18) (DeepSeek-R1, 17) (Claude 3.7, 13)};
					\end{axis}
				\end{tikzpicture}
			\end{center}
		\end{column}
		\begin{column}{0.48\textwidth}
			\begin{block}{Learning Progression}
				\begin{itemize}
					\item Initial success rate: 5\%
					\item After 1,000 steps: 22\%
					\item After 3,000 steps: 41\%
					\item Final accuracy: 53\%
					\item Model learned generalizable patterns beyond the training examples
				\end{itemize}
			\end{block}
		\end{column}
	\end{columns}
	\begin{block}{Key Success Factors}
		\begin{itemize}
			\item \textbf{Reward Design}: Multi-level rewards provided clear learning signals
			\item \textbf{Test Variety}: Diverse test cases prevented overfitting
			\item \textbf{Anti-Reward Hacking}: Prevented the model from simply memorizing outputs
		\end{itemize}
	\end{block}
\end{frame}

\section{Practical RFT Implementation with Unsloth}

\begin{frame}
	\frametitle{Hands-On RFT Workflow with Unsloth}
	\begin{block}{What is Unsloth?}
		\begin{itemize}
			\item Open-source library for efficient fine-tuning of large language models
			\item Enables fine-tuning even on limited hardware (3GB+ VRAM)
			\item Supports QLoRA, LoRA, RLHF, GRPO and other fine-tuning methods
			\item 2-4x faster training than standard implementations
		\end{itemize}
	\end{block}
	\begin{block}{Development Workflow}
		\begin{enumerate}
			\item \textbf{Setup Environment}: Install dependencies and import libraries
			\item \textbf{Select Model \& Method}: Choose base model and fine-tuning approach
			\item \textbf{Prepare Dataset}: Format data for training
			\item \textbf{Configure Training}: Set hyperparameters and training settings
			\item \textbf{Train \& Evaluate}: Run fine-tuning and monitor performance
			\item \textbf{Deploy}: Save and serve the fine-tuned model
		\end{enumerate}
	\end{block}
\end{frame}

\begin{frame}[fragile]
	\frametitle{Step 1: Environment Setup}
	\begin{block}{Install Unsloth}
		\begin{mintedbox}{bash}
# Basic installation
pip install unsloth

# For specific CUDA versions
# pip install unsloth[cu118]  # CUDA 11.8
# pip install unsloth[cu121]  # CUDA 12.1
		\end{mintedbox}
	\end{block}
\end{frame}

\begin{frame}[fragile]
	\frametitle{Step 1: Required Libraries}
	\begin{mintedbox}{python}
# Import necessary libraries
import torch
from unsloth import FastLanguageModel
from datasets import load_dataset
from transformers import TrainingArguments
	\end{mintedbox}
	\begin{block}{Performance Optimizations}
		\begin{itemize}
			\item CPU offloading for devices with limited VRAM
			\item Flash Attention 2 for faster training (on supported GPUs)
			\item Gradient checkpointing to trade compute for memory
		\end{itemize}
	\end{block}
\end{frame}

\begin{frame}[fragile]
	\frametitle{Step 2: Selecting Model \& Method}
	\begin{columns}
		\begin{column}{0.48\textwidth}
			\begin{block}{Load Base Model}
			\begin{mintedbox}{python}
max_seq_length = 2048
model, tokenizer = FastLanguageModel.from_pretrained(
    model_name = "unsloth/llama-3.1-8b-bnb-4bit",
    max_seq_length = max_seq_length,
    load_in_4bit = True)
			\end{mintedbox}
			\end{block}
		\end{column}
		\begin{column}{0.48\textwidth}
			\begin{block}{Choose Fine-Tuning Method}
				\begin{itemize}
					\item \textbf{QLoRA}: 4-bit quantization with LoRA (least resources)
					\item \textbf{LoRA}: Low-Rank Adaptation (balance of quality/resources)
					\item \textbf{GRPO}: For DeepSeek-style reinforcement learning
					\item \textbf{Full Finetuning}: For maximum quality on high-end hardware
				\end{itemize}
			\end{block}
		\end{column}
	\end{columns}
\end{frame}

\begin{frame}[fragile]
	\frametitle{Step 2: Configuring LoRA for RFT}
	\begin{block}{RFT-Specific Configuration}
		\begin{itemize}
			\item Target key-value attention layers for most efficient adaptation
			\item Use rank 16-32 for complex tasks like reinforcement learning
			\item Add dropout to prevent overfitting to reward signals
			\item Enable gradient checkpointing to save memory during training
		\end{itemize}
	\end{block}
	\begin{block}{Code Implementation}
		\begin{mintedbox}{python}
model = FastLanguageModel.get_peft_model(
    model,
    r = 16,  # LoRA rank
    target_modules = ["q_proj", "k_proj", "v_proj", "o_proj"],
    alpha = 16,  # LoRA alpha
    dropout = 0.05,  # Add regularization
    use_gradient_checkpointing = True)
		\end{mintedbox}
	\end{block}
\end{frame}

\begin{frame}[fragile]
	\frametitle{Step 3: Dataset Preparation}
	\begin{block}{Dataset Format for RFT}
		\begin{itemize}
			\item Format datasets with input and expected output
			\item Include reward signals or success criteria
			\item Ensure diverse examples across different scenarios
		\end{itemize}
	\end{block}
	\begin{mintedbox}{python}
dataset = [
    {"input": "Solve: $2x + 3 = 7$", 
     "output": "To solve for x:\n$2x + 3 = 7$\n$2x = 4$\n$x = 2$"},
    {"input": "What is the capital of France?", 
     "output": "The capital of France is Paris."},
    # ...more examples
]
	\end{mintedbox}
\end{frame}

\begin{frame}[fragile]
	\frametitle{Step 3: Dataset Formatting for Training}
	\begin{block}{Applying the Proper Template}
		\begin{itemize}
			\item Use the model's specific chat template format
			\item Ensure special tokens are correctly applied
			\item Set appropriate sequence length for your use case
		\end{itemize}
	\end{block}
	\begin{mintedbox}{python}
formatted_dataset = FastLanguageModel.format_dataset(
    dataset,
    tokenizer,
    max_seq_length,
    add_special_tokens = True,
    template = "<s>[INST] {input} [/INST] {output} </s>"
)
	\end{mintedbox}
\end{frame}

\begin{frame}[fragile]
	\frametitle{Step 4: Configure Training Parameters}
	\begin{block}{Training Arguments}
	\begin{mintedbox}{python}
training_args = TrainingArguments(
    output_dir = "./results",
    num_train_epochs = 3,
    per_device_train_batch_size = 4,
    gradient_accumulation_steps = 2,
    learning_rate = 2e-4,
    weight_decay = 0.01,
    max_grad_norm = 0.3,
    logging_steps = 10,
    save_total_limit = 3,
)
	\end{mintedbox}
	\end{block}
\end{frame}

\begin{frame}[fragile]
	\frametitle{Step 4: Hyperparameter Selection for RFT}
	\begin{columns}
		\begin{column}{0.48\textwidth}
			\begin{block}{Key Hyperparameters}
				\begin{itemize}
					\item \textbf{LoRA Rank (r)}: Between 8-32; higher gives more expressive power
					\item \textbf{LoRA Alpha}: Usually same as rank
					\item \textbf{Learning Rate}: 2e-4 to 5e-4 for QLoRA
					\item \textbf{Training Epochs}: 2-5 for small datasets
					\item \textbf{Dropout}: 0.05-0.1 for regularization
				\end{itemize}
			\end{block}
		\end{column}
		\begin{column}{0.48\textwidth}
			\begin{block}{Avoid Overfitting in RFT}
				\begin{itemize}
					\item Increase dropout if showing signs of overfitting
					\item Implement early stopping by monitoring loss
					\item Use a validation set to evaluate generalization
					\item Add weight decay for regularization
				\end{itemize}
			\end{block}
		\end{column}
	\end{columns}
\end{frame}

\begin{frame}[fragile]
	\frametitle{Step 5: Training Execution}
	\begin{block}{Execute Training}
		\begin{mintedbox}{python}
from trl import SFTTrainer

trainer = SFTTrainer(
    model = model,
    train_dataset = formatted_dataset,
    args = training_args,
    tokenizer = tokenizer,
    packing = False,
    dataset_text_field = "text",
)
		\end{mintedbox}
	\end{block}
\end{frame}

\begin{frame}[fragile]
	\frametitle{Step 5: Training Process and Monitoring}
	\begin{block}{Run Training}
		\begin{mintedbox}{python}
# Start training
trainer.train()
		\end{mintedbox}
	\end{block}
	\begin{block}{Monitor Key Metrics}
		\begin{itemize}
			\item \textbf{Training Loss}: Should steadily decrease but not plateau too quickly
			\item \textbf{Validation Loss}: Monitor for signs of overfitting (uptick in validation loss)
			\item \textbf{Learning Rate}: Record effect of LR variations on performance
			\item \textbf{GPU Utilization}: Verify efficient resource usage
		\end{itemize}
	\end{block}
\end{frame}

\begin{frame}[fragile]
	\frametitle{Step 6: Evaluation}
	\begin{block}{Evaluate Model Performance}
		\begin{itemize}
			\item Load the fine-tuned model for inference
			\item Create structured test prompts
			\item Compare with baseline model on same inputs
			\item Test diverse scenarios and edge cases
			\item Measure inference speed and resource usage
		\end{itemize}
	\end{block}
	\begin{mintedbox}{python}
# Example test inference
prompt = "Solve: 3x + 5 = 20"
response = model.generate_text(prompt)
# Result: "To solve: 3x + 5 = 20
Subtract 5: 3x = 15
Divide by 3: x = 5"
	\end{mintedbox}
\end{frame}

\begin{frame}[fragile]
	\frametitle{Step 6: Deployment}
	\begin{block}{Save and Deploy Model}
		\begin{itemize}
			\item Save trained model and tokenizer
			\item Optimize for inference (e.g., ONNX conversion) 
			\item Configure deployment environment
			\item Set up monitoring and logging
		\end{itemize}
	\end{block}
	\begin{mintedbox}{python}
# Save the fine-tuned model
model.save_pretrained("./my-rft-model")
tokenizer.save_pretrained("./my-rft-model")
	\end{mintedbox}
\end{frame}

\begin{frame}
	\frametitle{Step 6: Continuous Improvement}
	\begin{block}{Ongoing Refinement}
		\begin{itemize}
			\item Collect user feedback and model outputs in production
			\item Update reward functions based on real-world performance
			\item Expand training dataset with edge cases discovered in deployment
			\item Periodically retrain to incorporate improvements
		\end{itemize}
	\end{block}
	\begin{block}{Long-term Maintenance}
		\begin{itemize}
			\item Monitor for performance degradation over time
			\item Keep track of new best practices in RFT 
			\item Evaluate benefits of updating to newer base models
		\end{itemize}
	\end{block}
\end{frame}

\begin{frame}
	\frametitle{Ready-to-Use Unsloth Notebooks}
	\begin{block}{Available Implementation Resources}
		\begin{itemize}
			\item Unsloth provides ready-to-use notebooks for various models and tasks
			\item Accessible via Google Colab or Kaggle (free GPU resources)
			\item Includes both fine-tuning and GRPO (RFT) implementations
		\end{itemize}
	\end{block}
	\begin{columns}
		\begin{column}{0.48\textwidth}
			\begin{block}{Popular Models}
				\begin{itemize}
					\item Llama 3.1 (8B)
					\item Phi-4 (14B)
					\item Mistral (7B, 22B)
					\item Qwen 2.5 (3B, 14B)
					\item Gemma 2 (2B, 9B)
				\end{itemize}
			\end{block}
		\end{column}
		\begin{column}{0.48\textwidth}
			\begin{block}{Specialized Variants}
				\begin{itemize}
					\item Qwen2.5-Coder (14B)
					\item CodeGemma (7B)
					\item Llama 3.2 Vision
					\item Qwen2-VL (7B)
					\item Phi-3 Medium
				\end{itemize}
			\end{block}
		\end{column}
	\end{columns}
	\begin{center}
		\small{Notebooks available at: \texttt{https://docs.unsloth.ai/get-started/unsloth-notebooks}}
	\end{center}
\end{frame}

\begin{frame}
	\frametitle{Dataset Construction for RFT}
	\begin{block}{Getting Started}
		\begin{itemize}
			\item Identify the purpose of your dataset: chat dialogues, structured tasks, or domain-specific data
			\item Define desired output style: JSON, HTML, text, code or specific languages
			\item Determine data sources: Hugging Face datasets, Wikipedia, or synthetic data
		\end{itemize}
	\end{block}
	\begin{columns}
		\begin{column}{0.48\textwidth}
			\begin{block}{Common Data Formats}
				\begin{itemize}
					\item Text-only format
					\item Instruction-Input-Output
					\item ShareGPT format (multi-turn)
					\item ChatML (OpenAI style)
				\end{itemize}
			\end{block}
		\end{column}
		\begin{column}{0.48\textwidth}
			\begin{block}{Dataset Requirements}
				\begin{itemize}
					\item Minimum: 100 examples
					\item Optimal: 1,000+ examples
					\item Quality over quantity
					\item Can combine multiple datasets
				\end{itemize}
			\end{block}
		\end{column}
	\end{columns}
\end{frame}

\begin{frame}
	\frametitle{RFT Implementation: Common Pitfalls (1)}
	\begin{itemize}
		\item \textbf{Problem}: Model produces generic or refuses to generate responses\\
		  \textbf{Solution}: Increase LoRA rank and alpha; ensure diverse training data
		\item \textbf{Problem}: Catastrophic forgetting (model loses pre-trained capabilities)\\
		  \textbf{Solution}: Lower learning rate; add regularization
		\item \textbf{Problem}: High training loss that doesn't converge\\
		  \textbf{Solution}: Check dataset formatting; reduce sequence length
	\end{itemize}
\end{frame}

\begin{frame}
	\frametitle{RFT Implementation: Common Pitfalls (2)}
	\begin{itemize}
		\item \textbf{Problem}: Out of memory errors during training\\
		  \textbf{Solution}: Enable gradient checkpointing; reduce batch size
		\item \textbf{Problem}: Model generates hallucinations\\
		  \textbf{Solution}: Implement reward functions that penalize fabrications
		\item \textbf{Problem}: Training is unstable with reward signals\\
		  \textbf{Solution}: Normalize rewards; use PPO with appropriate clipping
	\end{itemize}
\end{frame}

\section{Conclusion}

\begin{frame}
	\frametitle{Key Takeaways}
	\begin{block}{The Power of Reinforcement Fine-Tuning:}
		\begin{itemize}
			\item \textbf{Self-Improvement is Real}: Models can surpass static approaches through iterative rewards
			\item \textbf{RFT Shines With Little Data}: Outperforms SFT when labeled data is scarce
			\item \textbf{Reasoning Doesn't Need to be Slow}: Turbo LoRA increases throughput by 2-4x
			\item \textbf{Real-World Feasibility}: Moving beyond academic exercises to practical applications
		\end{itemize}
	\end{block}
	\begin{block}{Paradigm Shift in AI Development}
		\begin{itemize}
			\item Traditional LLM training: massive datasets $\rightarrow$ static deployment
			\item RFT approach: minimal datasets + reward functions $\rightarrow$ continuous improvement
			\item Fundamentally changes how we think about model improvement cycles
			\item Smaller teams can now compete with resource-rich organizations
		\end{itemize}
	\end{block}
\end{frame}

\begin{frame}
	\frametitle{Implementation Strategy - Core Components}
	\begin{block}{Core Components Needed}
		\begin{enumerate}
			\item Base model (open-source LLM)
			\item Reward function definition
			\item Prompt dataset (can be small)
			\item RL algorithm (RLHF, GRPO, etc.)
			\item Evaluation framework
		\end{enumerate}
	\end{block}
\end{frame}

\begin{frame}
	\frametitle{Implementation Strategy - Best Practices}
	\begin{block}{Best Practices}
		\begin{itemize}
			\item Start with small, clear reward functions
			\item Build comprehensive validation tests
			\item Implement anti-reward-hacking measures
			\item Monitor for emergent behaviors
			\item Gradually increase complexity
		\end{itemize}
	\end{block}
\end{frame}

\begin{frame}
	\frametitle{Next Steps with RFT}
	\begin{block}{Getting Started:}
		\begin{itemize}
			\item Reinforcement fine-tuning marks a major leap in LLM development
			\item Training through reward signals rather than labeled examples
			\item End-to-end platforms make this approach accessible to developers
			\item Focus on innovation rather than infrastructure complexities
		\end{itemize}
	\end{block}
	\begin{block}{Areas Ripe for RFT Application}
		\begin{itemize}
			\item \textbf{Specialized Coding}: Domain-specific code generation (embedded systems, high-performance computing)
			\item \textbf{Scientific Research}: Models that propose and validate hypotheses
			\item \textbf{Reasoning Tasks}: Complex logical and mathematical problem-solving
			\item \textbf{Education}: Adaptive tutoring systems that understand student gaps
		\end{itemize}
	\end{block}
\end{frame}

\begin{frame}{}
	\begin{center}
		\LARGE Thank You!
	\end{center}
	\begin{center}
		\large Questions?
	\end{center}
\end{frame}

\end{document}
