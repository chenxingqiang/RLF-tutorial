% beamer幻灯片类,采用16:9宽高比
\documentclass[aspectratio=169]{beamer}

% 加载所有软件包
% input encoding
\usepackage[utf8]{inputenc}

%--------------------------------------------
% Font encoding
%--------------------------------------------
\usepackage[T1]{fontenc} % Schriftherkunft

%--------------------------------------------
% Code listings
%--------------------------------------------
\usepackage{listings}
\usepackage{xcolor}

% For better code display with tcolorbox and minted
\usepackage{tcolorbox}
\tcbuselibrary{minted,breakable,xparse,skins}
\usepackage{minted}

% Define colors for code boxes
\definecolor{bg}{gray}{0.95}

% Define beautiful code box - optimized for beamer slides
\DeclareTCBListing{mintedbox}{O{}m!O{}}{%
  breakable=true,
  listing engine=minted,
  listing only,
  minted language=#2,
  minted style=default,
  minted options={%
    linenos,
    gobble=0,
    breaklines=true,
    breakafter=,,
    fontsize=\footnotesize,
    numbersep=5pt,
    autogobble=true,
    #1},
  boxsep=0pt,
  left skip=0pt,
  right skip=0pt,
  left=20pt,
  right=0pt,
  top=2pt,
  bottom=2pt,
  arc=3pt,
  leftrule=0pt,
  rightrule=0pt,
  bottomrule=1pt,
  toprule=1pt,
  colback=bg,
  colframe=orange!70,
  width=\textwidth-10pt,
  nobeforeafter=true,
  size=minimal,
  enhanced,
  overlay={%
    \begin{tcbclipinterior}
    \fill[orange!20!white] (frame.south west) rectangle ([xshift=15pt]frame.north west);
    \end{tcbclipinterior}},
  #3,
}

% Basic listings settings (for compatibility)
\lstset{
  basicstyle=\ttfamily\small,
  keywordstyle=\color{blue},
  commentstyle=\color{gray},
  breaklines=true,
  showstringspaces=false,
  frame=single,
  frameround=tttt
}

% package for the English language
\usepackage[english]{babel}

% default graphics path
\graphicspath{{logos/}{figures/}}

% package to create diagrams and plot with TikZ
\usepackage{pgfplots}
% always use the newest version
\pgfplotsset{compat=newest}
% define the style of the connection of two lines
\pgfplotsset{every axis/.append style={line join=bevel}}
% generate a new style template
\mode<beamer>{
	\pgfplotsset{
		beamer/.style={
			width=0.8\textwidth,
			height=0.45\textwidth,
			legend style={font=\scriptsize},
			tick label style={font=\footnotesize},
			label style={font=\small},
			max space between ticks=28,
		}
	}
}
\mode<handout>{
	\pgfplotsset{
		beamer/.style={
			width=0.8\textwidth,
			height=0.45\textwidth,
			legend style={font=\scriptsize},
			tick label style={font=\footnotesize},
			label style={font=\small},
			max space between ticks=25,
		}
	}
}
\mode<article>{
	\pgfplotsset{
		beamer/.style={
			width=0.8\textwidth,
			height=0.45\textwidth,
			max space between ticks=35,
		}
	}
}
% format and size template for two plots side-by-side
\pgfplotsset{
	scriptsize/.style={
		width=0.34\textwidth,
		height=0.1768\textwidth,
		legend style={font=\scriptsize},
		tick label style={font=\scriptsize},
		label style={font=\footnotesize},
		title style={font=\footnotesize},
		every axis title shift=0pt,
		max space between ticks=25,
		every mark/.append style={mark size=7},
		major tick length=0.1cm,
		minor tick length=0.066cm,
	}
}
\pgfplotsset{
	small/.style={
		width=6.5cm,
		height=,
		tick label style={font=\footnotesize},
		label style={font=\small},
		legend style={font=\footnotesize},
		max space between ticks=30,
	}
}
% align legends entries to the left
\pgfplotsset{legend cell align=left}
% draw a main grid
\pgfplotsset{xmajorgrids}
\pgfplotsset{ymajorgrids}
% number of minor ticks between two major tickmarks
%\pgfplotsset{minor x tick num={3}}
%\pgfplotsset{minor y tick num={3}}
% draw a minor grid
%\pgfplotsset{xminorgrids}
%\pgfplotsset{yminorgrids}
% scale only the axis using a certain width and height
\pgfplotsset{scale only axis}
% define colors like in MATLAB
\definecolor{matlab1}{rgb}{0,0,1}
\definecolor{matlab2}{rgb}{0,0.5,0}
\definecolor{matlab3}{rgb}{1,0,0}
\definecolor{matlab4}{rgb}{0,0.75,0.75}
\definecolor{matlab5}{rgb}{0.75,0,0.75}
\definecolor{matlab6}{rgb}{0.75,0.75,0}
\definecolor{matlab7}{rgb}{0.25,0.25,0.25}
% define a color cycle list like in MATLAB
\pgfplotscreateplotcyclelist{matlab}{
	{matlab1,solid},
	{matlab2,dashed},
	{matlab3,dashdotted},
	{matlab4,dotted},
	{matlab5,densely dashed},
	{matlab6,densely dashdotted},
	{matlab7,densely dotted}%this prevents an error
}
% use the earlier defined color list
\pgfplotsset{cycle list name=matlab}
% use the standard color list of pgfplots
%\pgfplotsset{cycle list name=color list}
% use only grayscale lines
%\pgfplotsset{cycle list name=linestyles}
% use a line width of 1pt
\pgfplotsset{every axis plot/.append style={line width=1pt}}
% use a comma as the dot separator for german documents
\addto\extrasngerman{\pgfplotsset{/pgf/number format/.cd,set decimal separator={{{,}}}}}
% use a half space as the 1000 separator
\pgfplotsset{/pgf/number format/.cd,1000 sep={\,}}
% define legend positions
\pgfplotsset{/pgfplots/legend pos/north/.style={/pgfplots/legend style={at={(0.50,0.97)},anchor=north}}}
\pgfplotsset{/pgfplots/legend pos/south/.style={/pgfplots/legend style={at={(0.50,0.03)},anchor=south}}}
\pgfplotsset{/pgfplots/legend pos/east/.style={/pgfplots/legend style={at={(0.97,0.50)},anchor=east}}}
\pgfplotsset{/pgfplots/legend pos/west/.style={/pgfplots/legend style={at={(0.03,0.50)},anchor=west}}}
\pgfplotsset{/pgfplots/legend pos/outer north/.style={/pgfplots/legend style={at={(0.50,1.03)},anchor=south}}}

% package for SI units
\usepackage{siunitx}
% real fractions
\sisetup{per-mode=fraction,mode=math}
% decimal marker in dependence from the language
\addto\extrasngerman{\sisetup{output-decimal-marker={,}}}
\addto\extrasenglish{\sisetup{output-decimal-marker={.}}}
% range phrase in dependence from the language
\addto\extrasngerman{\sisetup{range-phrase={ bis~}}} 
\addto\extrasenglish{\sisetup{range-phrase={ to~}}}

% A stand-alone package that implements several commands for including external animation and sound
% files in a pdf document. The package can be used together with both dvips plus ps2pdf and pdflatex,
% though the special sound support is available only in pdflatex.
\usepackage{multimedia}

% package to dissable floating in the article mode
\usepackage{float}

% package to change the line spacing
\usepackage{setspace}

% package for nice fractions in the text mode
\usepackage{xfrac}
% Comment out the problematic command
% \UseCollection{xfrac}{plainmath}

% nicer tables
\usepackage{booktabs}

% nicer quotation marks
\usepackage{csquotes}

% define labeling environment
\makeatletter
\newenvironment{labeling}[2][]{%
  \def\sc@septext{#1}%
  \list{}{\settowidth{\labelwidth}{{%
        #2%
          \sc@septext%
      }}%
    \leftmargin\labelwidth \advance\leftmargin by \labelsep
    \let\makelabel\labelinglabel
  }%
}{%
  \endlist
}
\newcommand\labelinglabel[1]{%
  #1\hfil
    \sc@septext%
}
\makeatother

% 暂时注释掉BibLaTeX设置,因为没有参考文献需要引用
% \usepackage[%
% 	style=ieee,%				format as used at the IEEE
% 	backend=biber,%			suppress warning
% ]{biblatex}

% database of references
% \addbibresource{literature.bib}

% How to remove some pages from navigation bullets in beamer?
% see: https://tex.stackexchange.com/questions/37127/how-to-remove-some-pages-from-the-navigation-bullets-in-beamer
\makeatletter
\let\beamer@writeslidentry@miniframeson=\beamer@writeslidentry
\def\beamer@writeslidentry@miniframesoff{%
	\expandafter\beamer@ifempty\expandafter{\beamer@framestartpage}{}% does not happen normally
	{%else
		% removed \addtocontents commands
		\clearpage\beamer@notesactions%
	}
}
\newcommand*{\miniframeson}{\let\beamer@writeslidentry=\beamer@writeslidentry@miniframeson}
\newcommand*{\miniframesoff}{\let\beamer@writeslidentry=\beamer@writeslidentry@miniframesoff}
\makeatother

% Distance between points of subsections in miniframe navigation
% siehe: http://tex.stackexchange.com/questions/288548/beamer-miniframes-horizontal-space-between-dots-when-using-the-compress-option
\makeatletter
\def\slideentry#1#2#3#4#5#6{%
  %section number, subsection number, slide number, first/last frame, page number, part number
  \ifnum#6=\c@part\ifnum#2>0\ifnum#3>0%
    \ifbeamer@compress%
      \advance\beamer@xpos by1\relax%
      \ifnum1=#3                        %%% NOTICE
				\ifnum1=#2\else
					\advance\beamer@xpos by1\relax% %%% THESE
				\fi
      \fi                               %%% LINE
    \else%
      \beamer@xpos=#3\relax%
      \beamer@ypos=#2\relax%
    \fi%
  \hbox to 0pt{%
    \beamer@tempdim=-\beamer@vboxoffset%
    \advance\beamer@tempdim by-\beamer@boxsize%
    \multiply\beamer@tempdim by\beamer@ypos%
    \advance\beamer@tempdim by -.05cm%
    \raise\beamer@tempdim\hbox{%
      \beamer@tempdim=\beamer@boxsize%
      \multiply\beamer@tempdim by\beamer@xpos%
      \advance\beamer@tempdim by -\beamer@boxsize%
      \advance\beamer@tempdim by 1pt%
      \kern\beamer@tempdim
      \global\beamer@section@min@dim\beamer@tempdim
      \hbox{\beamer@link(#4){%
          \usebeamerfont{mini frame}%
          \ifnum\c@section=#1%
            \ifnum\c@subsection=#2%
              \usebeamercolor[fg]{mini frame}%
              \ifnum\c@subsectionslide=#3%
                \usebeamertemplate{mini frame}%\beamer@minislidehilight%
              \else%
                \usebeamertemplate{mini frame in current subsection}%\beamer@minisliderowhilight%
              \fi%
            \else%
              \usebeamercolor{mini frame}%
              %\color{fg!50!bg}%
              \usebeamertemplate{mini frame in other subsection}%\beamer@minislide%
            \fi%
          \else%
            \usebeamercolor{mini frame}%
            %\color{fg!50!bg}%
            \usebeamertemplate{mini frame in other subsection}%\beamer@minislide%
          \fi%
        }}}\hskip-10cm plus 1fil%
  }\fi\fi%
  \else%
  \fakeslideentry{#1}{#2}{#3}{#4}{#5}{#6}%
  \fi\ignorespaces
}


% 添加中文支持
\usepackage{xeCJK}
\usepackage{ctex}
\setCJKmainfont{STSong}
\setCJKsansfont{STHeiti}
\setCJKmonofont{STFangsong}

% 确保所有字体都能正确显示中文
\newfontfamily\zhfont{STSong}
\newfontfamily\zhsffont{STHeiti}
\newfontfamily\zhttfont{STFangsong}

% 设置beamer字体
\usefonttheme{professionalfonts}
\setbeamerfont{title}{family=\zhsffont}
\setbeamerfont{frametitle}{family=\zhsffont}
\setbeamerfont{framesubtitle}{family=\zhsffont}

% 使用修改后的OVGU主题,带有之江实验室标志
\usepackage{beamer_ovgu_169}

% 全局设置以防止内容溢出
\usepackage{pgfplots}
\usetikzlibrary{patterns}

% 优化幻灯片布局以防止溢出
\setbeamersize{text margin left=0.5cm, text margin right=0.5cm}
\setbeamertemplate{frametitle}[default][center]

% 设置全局框架选项以自动处理溢出
\makeatletter
\define@key{beamerframe}{shrink}[100]{%
  \def\beamer@shrinkpercentage{#1}}
\makeatother
\BeforeBeginEnvironment{frame}{% 
  \setkeys{beamerframe}{shrink=45}% 
}

% 全局设置以处理中文内容溢出问题
\addtobeamertemplate{frametitle}{}{\vspace{-0.2cm}}
\addtobeamertemplate{block begin}{}{\vspace{-0.2cm}}
\addtobeamertemplate{itemize item}{}{\vspace{-0.1cm}}

% 对长框架的额外调整
\newenvironment{compactframe}{
  \begin{frame}[shrink=60]
  \setbeamerfont{itemize/enumerate body}{size=\scriptsize}
  \setbeamerfont{itemize/enumerate subbody}{size=\scriptsize}
}{\end{frame}}

% Adjust spacing throughout the presentation
\addtobeamertemplate{frametitle}{}{\vspace{-0.3cm}}
\setlength{\itemsep}{0pt}
\setlength{\parskip}{0pt}
\setlength{\parsep}{0pt}
\setlength{\partopsep}{0pt}

% Adjust beamer spacing settings to be more compact
\AtBeginDocument{
  \setbeamerfont{itemize/enumerate body}{size=\small}
  \setbeamerfont{itemize/enumerate subbody}{size=\small}
  \setbeamerfont{itemize/enumerate subsubbody}{size=\small}
}

% 禁用参考文献功能,避免bbl文件缺失警告
\providecommand{\bibfont}{}
\renewcommand{\bibfont}{\normalfont\footnotesize}
\AtEndDocument{\def\bibfont{}}
\let\printbibliography\relax

\title[强化微调指南 2025]{强化微调完全指南}
\author{陈星强}
\institute[亿铸智能]{
	亿铸智能
    \\
	中国杭州
}
\date[2025-04-25]{2025年4月25日}

\begin{document}

\begin{frame}
	\maketitle
\end{frame}

\begin{frame}[label=inhalt]{内容概述}
	\tableofcontents
\end{frame}

\section{使用DeepSeek-R1和强化学习重塑AI}

\begin{frame}
	\frametitle{强化学习简介}
	\begin{block}{自我提升范式}
		\begin{itemize}
			\item 强化学习引入了一种基于反馈驱动的AI训练机制
			\item 相比依赖标记样本,强化学习代理通过以下方式学习:
			\begin{itemize}
				\item \textbf{探索}:模型尝试多种策略或行动
				\item \textbf{奖励}:每个行动产生指导未来选择的奖励信号
			\end{itemize}
			\item 更接近人类通过尝试、错误和反馈自然学习的方式
		\end{itemize}
	\end{block}
\end{frame}

\begin{frame}
	\frametitle{强化学习简介 - 继续}
	\begin{block}{自我提升范式(继续)}
		\begin{itemize}
			\item 为持续学习和更深层的推理能力打开了大门
			\item AI不再是静态的实践—模型可以在部署后进化和适应
		\end{itemize}
	\end{block}
	\begin{block}{从静态到动态学习}
		\begin{itemize}
			\item 传统方法依赖大量标记数据集进行记忆
			\item 强化学习将焦点转移到学习策略和推理模式
			\item “环境”可以是任何提供反馈信号指导改进的场景
		\end{itemize}
	\end{block}
\end{frame}

\begin{frame}
	\frametitle{DeepSeek-R1的重要性}
	\begin{block}{DeepSeek-R1:一个为推理而设计的模型}
		\begin{itemize}
			\item \textbf{自适应奖励结构}:多重奖励函数聚焦于准确性、效率性和创造性
			\item \textbf{迭代精化}:基于奖励的反馈循环强调实践中最有效的方法
			\item \textbf{突破性能障碍}:持续学习使其能够超越传统大语言模型
			\item \textbf{开源优势}:与OpenAI闭源的O1不同,DeepSeek-R1共享模型权重和训练方法
		\end{itemize}
	\end{block}
\end{frame}

\begin{frame}[shrink=10]
	\frametitle{\textbf{DeepSeek-R1}: 技术实现}
	\begin{columns}[T]
		\begin{column}{0.48\textwidth}
			\begin{block}{主动探索机制}
				\begin{itemize}\setlength{\itemsep}{0em}
					\item 使用\textbf{主动探索}而非被动学习
					\item 采用\textbf{基于奖励的反馈循环}
					\item 平衡探索与利用
				\end{itemize}
			\end{block}
		\end{column}
		\begin{column}{0.48\textwidth}
			\begin{block}{推理模型民主化}
				\begin{itemize}\setlength{\itemsep}{0em}
					\item \textbf{开源设计}便于自定义
					\item \textbf{发布模型权重}保证透明度
					\item \textbf{社区驱动}的创新
				\end{itemize}
			\end{block}
		\end{column}
	\end{columns}
\end{frame}

\begin{frame}[shrink=20]
	\frametitle{DeepSeek-R1 与传统模型对比}
	\vspace{-0.2cm}
	\begin{block}{关键差异:}
		\setlength{\itemsep}{0.5em}
		\begin{itemize}
			\item \textbf{传统模型:} 静态训练,固定参数
			\item \textbf{DeepSeek-R1:} 动态学习方法
		\end{itemize}
	\end{block}
\end{frame}

\begin{frame}
	\frametitle{DeepSeek-R1 与传统模型对比 (继续)}
	\begin{block}{DeepSeek-R1 优势:}
		\begin{itemize}
			\item 动态学习,响应需求变化
			\item 通过奖励学习,减少对标记数据的依赖
			\item 通过持续学习降低停滞风险
		\end{itemize}
	\end{block}
	\begin{block}{关键见解:}
		\begin{itemize}
			\item “一次性训练完成”模式正在成为过去式
		\end{itemize}
	\end{block}
\end{frame}

\section{强化微调与监督微调对比}

\begin{frame}
	\frametitle{什么是强化微调(RFT)?}
	\begin{block}{将微调与强化学习相结合}
		\begin{itemize}
			\item RFT结合了预训练大语言模型与基于反馈的强化学习的优势
			\item 核心过程:
			\begin{enumerate}
				\item 从具有通用知识的预训练模型开始
				\item 为目标指标定义奖励函数
				\item 使用强化学习技术进行迭代微调
			\end{enumerate}
			\item 使用最少的数据实现开源大语言模型的自定义
			\item 将通用模型转变为针对特定任务的强大推理模型
		\end{itemize}
	\end{block}
\end{frame}

\begin{frame}
	\frametitle{RFT与SSFT对比}
	\begin{table}
		\footnotesize
		\begin{tabular}{|p{3cm}|p{3cm}|p{3cm}|}
			\hline
			\textbf{因素} & \textbf{RFT} & \textbf{SFT} \\
			\hline
			Data Requirements & Minimal labeled data & Needs 1,000+ rows \\
			\hline
			Adaptability & Continuous improvement & Limited by labeled data \\
			\hline
			Exploration & Actively tries new strategies & Relies on fixed examples \\
			\hline
			Performance & Continual progress & Reaches plateau \\
			\hline
			Error Handling & Learns from mistakes & Repeats errors in data \\
			\hline
			Training Complexity & Higher (reward function) & Lower (just examples) \\
			\hline
		\end{tabular}
		\caption{Comparison of RFT and SFT approaches}
	\end{table}
\end{frame}

\begin{frame}
	\frametitle{When RFT Wins}
	\begin{block}{RFT Excels with Scarce Data}
		\begin{itemize}
			\item Removes need for labeled data, relies on objective correctness
			\item Outperforms SFT with small datasets (dozens of examples)
			\item Resists overfitting by learning robust strategies
			\item Best use cases:
			\begin{itemize}
				\item Code Transpilation (e.g., Java to Python)
				\item Game Strategy (Chess, Wordle)
				\item Medical Diagnosis (learning from feedback at decision points)
			\end{itemize}
		\end{itemize}
	\end{block}
\end{frame}

\begin{frame}
	\frametitle{Quantitative Performance Comparison}
	\begin{block}{RFT vs SFT by Dataset Size}
		\begin{itemize}
			\item \textbf{10 Examples}: RFT improved base model by 18\%, SFT showed minimal gains
			\item \textbf{50 Examples}: RFT showed 42\% improvement over baseline
			\item \textbf{100 Examples}: RFT improvement jumped to 60\%, 3x better than SFT
			\item 数据集越小,RFT相比SFT的优势越明显
		\end{itemize}
	\end{block}
	\begin{figure}[ht]
		\centering
		\begin{tikzpicture}[scale=0.5, transform shape]
			\begin{axis}[
				xlabel={数据集大小},
				ylabel={性能提升 (\%)},
				legend style={at={(0.5,-0.15)}, anchor=north},
				ymin=0,
				ymax=70,
				grid=both,
				minor tick num=1,
				width=8cm,
				height=5cm,
				legend columns=2,
				scaled ticks=false,
				font=\footnotesize,
				xtick={10, 50, 100}]
				\addplot[color=blue, mark=*,line width=1.2pt] coordinates {
					(10,18) (50,42) (100,60)
				};
				\addplot[color=red, mark=square*,line width=1.2pt] coordinates {
					(10,5) (50,15) (100,20)
				};
				\legend{RFT, SFT};
			\end{axis}
		\end{tikzpicture}
		\caption{性能提升与数据集大小的关系}
	\end{figure}
\end{frame}

\begin{frame}[shrink=10]
	\frametitle{何时使用RFT与SFT}
	\begin{block}{决策因素:}
		\begin{itemize}
			\item \textbf{数据可用性}: 数据有限时选择RFT;有大量标记数据时选择SFT
			\item \textbf{任务复杂性}: 对于有明确成功标准的任务,选择RFT
			\item \textbf{性能目标}: 需要持续改进时选择RFT;需要稳定结果时选择SFT
			\item \textbf{可验证性}: 当结果可以客观衡量时,RFT表现更佳
			\item \textbf{资源限制}: SFT初期实施更简单,但需要更多数据
		\end{itemize}
	\end{block}
	\begin{block}{实际应用:}
		\begin{itemize}
			\item \textbf{编程助手}: RFT训练模型编写可编译并通过测试的代码
			\item \textbf{数据分析}: RFT改进生成准确结果的查询能力
			\item \textbf{推理任务}: RFT增强逐步解决问题的能力
		\end{itemize}
	\end{block}
\end{frame}

\section{使用Turbo LoRA加速推理模型}

\begin{frame}[shrink=10]
	\frametitle{推理模型的挑战}
	\begin{block}{处理量问题:}
		\begin{itemize}
			\item 推理模型推动了AI的解决问题能力
			\item 但高级推理需要付出代价:处理速度缓慢
			\item 推理模型通过生成大量标记来“思考”
			\item 多重中间计算增加了处理时间
			\item 生产部署需要解决这些挑战
		\end{itemize}
	\end{block}
	\begin{block}{实际性能瓶颈}
		\begin{itemize}
			\item 推理模型比类似的非推理模型慢2-3倍
			\item 更高的延迟显著影响用户体验和成本效率
			\item 传统加速方法常常会损害推理质量
			\item 需要特殊解决方案,在提高速度的同时保持推理能力
		\end{itemize}
	\end{block}
\end{frame}

\begin{frame}[shrink=10]
	\frametitle{LoRA和Turbo LoRA}
	\begin{block}{LoRA: 低秩适应}
		\begin{itemize}
			\item 使用小型可训练参数集微调大型模型
			\item 保留原始权重,维持预训练知识
			\item 显著减少微调所需的内存要求
			\item 支持模型高效适应专业任务
		\end{itemize}
	\end{block}
\end{frame}

\begin{frame}[shrink=15]
	\frametitle{Turbo LoRA: 推理速度提升2-4倍}
	\begin{block}{主要特点}
		\begin{itemize}
			\item 使用推测解码以及专有优化
			\item 并行预测多个标记,然后验证它们
			\item 在更快生成文本的同时保持输出质量
		\end{itemize}
	\end{block}
\end{frame}

\begin{frame}[shrink=20]
	\frametitle{Turbo LoRA如何工作}
	\begin{columns}[T]
		\begin{column}{0.48\textwidth}
			\begin{block}{技术实现}
				\begin{enumerate}\setlength{\itemsep}{0.5em}
					\item 小型、快速的“推测器”模型并行预测多个标记
					\item 主模型验证预测的标记
					\item 正确的标记立即被使用
					\item 只有不正确的标记需要重新计算
				\end{enumerate}
			\end{block}
		\end{column}
		\begin{column}{0.48\textwidth}
			\begin{block}{实际效益}
				\begin{itemize}\setlength{\itemsep}{0.5em}
					\item 最终输出质量零差异
					\item 应用于DeepSeek-R1-distill-qwen-32b模型
					\item 实现了2-3倍的处理量提升
					\item 可应用于任何推理模型
				\end{itemize}
			\end{block}
		\end{column}
	\end{columns}
	\vspace{0.2cm}
	\begin{center}
		\resizebox{0.85\textwidth}{!}{
		\begin{tikzpicture}
			\draw[thick,->] (0,0) -- (9,0) node[right] {时间};
			% Standard generation
			\draw[thick] (0,1) -- (9,1);
			\foreach \x in {1,2,3,4,5,6,7,8}
				\draw[fill=gray!30] (\x,0.7) rectangle (\x+0.5,1.3) node[pos=.5] {$t_\x$};
			% Turbo generation
			\draw[thick] (0,2.5) -- (9,2.5);
			\draw[fill=blue!20] (1,2.2) rectangle (4,2.8) node[pos=.5] {$t_{1-4}$};
			\draw[fill=blue!20] (4.5,2.2) rectangle (7.5,2.8) node[pos=.5] {$t_{5-8}$};
			% Labels
			\node at (-1,1) {标准:};
			\node at (-1,2.5) {Turbo:};
		\end{tikzpicture}
		}
	\end{center}
\end{frame}

\begin{frame}[shrink=10]
	\frametitle{Turbo对大型推理模型的益处}
	\begin{block}{实现实时AI可行性}
		\begin{itemize}
			\item 2-3倍的速度提升使推理模型适用于:
			\begin{itemize}
				\item AI驱动的客户支持
			\end{itemize}
		\end{itemize}
	\end{block}
\end{frame}

\begin{frame}[shrink=10]
	\frametitle{Turbo对大型推理模型的益处(续)}
	\begin{block}{实现实时AI可行性(续)}
		\begin{itemize}
			\item 2-3倍的速度提升使推理模型适用于:
			\begin{itemize}
				\item 开发者的AI副驾驶
				\item 医疗保健AI助手
			\end{itemize}
			\item 降低GPU成本:相同工作量需要更少的GPU
		\end{itemize}
	\end{block}
\end{frame}

\begin{frame}[shrink=10]
	\frametitle{大型推理模型Turbo的优势(续)}
	\begin{block}{实际影响}
		\begin{itemize}
			\item 使几乎任何组织都能负担推理模型的运行
			\item 典型推理设置:所需GPU容量减少2-3倍
			\item 成本节约随部署规模扩大而增加
		\end{itemize}
	\end{block}
	\begin{block}{实施方法}
		\begin{itemize}
			\item 可使用更小、更高效的模型而不牺牲能力
			\item 详细实施教程: \texttt{https://predibase.com/blog/turbo-lora}
		\end{itemize}
	\end{block}
\end{frame}

\section{Tutorial: Using RFT to Write CUDA Kernels}

\begin{frame}[shrink=10]
	\frametitle{Why GPU Code Generation is Hard}
	\begin{columns}[T]
		\begin{column}{0.48\textwidth}
			\begin{block}{Challenges of GPU Coding:}
				\begin{itemize}\setlength{\itemsep}{0em}
					\item \textbf{Parallel Architecture}: Many cores
					\item \textbf{Memory Hierarchy}: Multiple levels
					\item \textbf{Thread Synchronization}: Complex
					\item Small errors → large consequences
				\end{itemize}
			\end{block}
		\end{column}
		\begin{column}{0.48\textwidth}
			\begin{block}{Traditional Approaches Fail:}
				\begin{itemize}\setlength{\itemsep}{0em}
					\item Few high-quality CUDA examples
					\item SFT needs thousands of pairs
					\item Many edge cases to handle
					\item Subtle errors cause failures
				\end{itemize}
			\end{block}
		\end{column}
	\end{columns}
\end{frame}

\mode<all>{\vspace{-0.5cm}}

\begin{frame}[t]
	\frametitle{Why RFT is Perfect for Code Generation}
	\begin{columns}[T]
		\begin{column}{0.48\textwidth}
			\begin{block}{Key Advantages:}
				\begin{itemize}
					\item No large dataset needed
					\item Code is verifiable
				\end{itemize}
			\end{block}
		\end{column}
		\begin{column}{0.48\textwidth}
			\begin{block}{Implementation:}
				\begin{itemize}
					\item Started with 13 examples
					\item RL explores intelligently
				\end{itemize}
			\end{block}
		\end{column}
	\end{columns}
\end{frame}

\begin{frame}
	\frametitle{Setting Up the CUDA Task: Minimal Dataset}
	\begin{block}{Dataset Composition}
		\begin{itemize}
			\item Each example contained:
			\begin{itemize}
				\item A PyTorch function (e.g., matrix multiply or activation function)
				\item A set of test cases to verify correctness
			\end{itemize}
			\item Example functions included matrix operations, activations, and element-wise operations
			\item Test cases covered edge cases, different sizes, and boundary conditions
		\end{itemize}
	\end{block}
	\begin{block}{PyTorch to Triton Example}
	\begin{itemize}
		\item \textbf{PyTorch function:}
		\begin{quote}
		\texttt{def add(x, y): return x + y}
		\end{quote}
	\end{itemize}
	\end{block}
\end{frame}

\begin{frame}
	\frametitle{Setting Up the CUDA Task: Triton Implementation}
	\begin{block}{Target Triton Kernel}
	\begin{quote}
	\texttt{@triton.jit}\\  
	\texttt{def add\_kernel(x\_ptr, y\_ptr, output\_ptr, n\_elements):}\\  
	\texttt{\phantom{xxxx}pid = tl.program\_id(0)}\\  
	\texttt{\phantom{xxxx}block\_size = 128}\\  
	\texttt{\phantom{xxxx}offsets = pid * block\_size + tl.arange(0, block\_size)}
	\end{quote}
	\end{block}
\end{frame}

\begin{frame}
	\frametitle{Setting Up the CUDA Task: Triton Implementation (cont.)}
	\begin{block}{Target Triton Kernel (cont.)}
	\begin{quote}
	\texttt{\phantom{xxxx}mask = offsets < n\_elements}\\  
	\texttt{\phantom{xxxx}x = tl.load(x\_ptr + offsets, mask=mask)}\\  
	\texttt{\phantom{xxxx}y = tl.load(y\_ptr + offsets, mask=mask)}\\  
	\texttt{\phantom{xxxx}output = x + y}\\  
	\texttt{\phantom{xxxx}tl.store(output\_ptr + offsets, output, mask=mask)}
	\end{quote}
	\end{block}
\end{frame}

\begin{frame}
	\frametitle{Defining Rewards for Code Generation}
	\begin{block}{Multi-Level Reward Structure (1)}
		\begin{description}
			\item[\textbf{Reward 1:}] \textbf{Formatting (0.1-0.3)}
		\end{description}
		\begin{itemize}
			\item Code structure, imports, tags
			\item Partial credit for good Triton semantics
			\item Reasonable variable names and code organization
		\end{itemize}
	\end{block}
	\begin{block}{Multi-Level Reward Structure (2)}
		\begin{description}
			\item[\textbf{Reward 2:}] \textbf{Compilation (0.3-0.6)}
		\end{description}
		\begin{itemize}
			\item Code that executes without errors
			\item No runtime exceptions or syntax errors
			\item Properly imports required dependencies
		\end{itemize}
	\end{block}
\end{frame}

\begin{frame}
	\frametitle{Defining Rewards for Code Generation (cont.)}
	\begin{block}{Multi-Level Reward Structure (3)}
		\begin{description}
			\item[\textbf{Reward 3:}] \textbf{Correctness (0.6-1.0)}
		\end{description}
	\end{block}
	\begin{itemize}
		\item Output matches PyTorch function on test inputs
		\item Anti-reward-hacking measures (checking for hardcoded outputs)
		\item Proper handling of edge cases and different input shapes
	\end{itemize}
	\vspace{0.5em}
	\begin{block}{Benefits of This Reward Structure}
		\begin{itemize}
			\item Provides clear progression path for the model
			\item Allows partial credit for partial solutions
			\item Mirrors how humans learn to code
		\end{itemize}
	\end{block}
\end{frame}

\begin{frame}
	\frametitle{Example Implementation of Reward Function}
	\begin{block}{Reward Function Logic - Part 1}
		\begin{enumerate}
			\item \textbf{Formatting check:}
			\begin{itemize}
				\item If code has proper Triton syntax $\rightarrow$ reward = 0.2
			\end{itemize}
			\item \textbf{Compilation check:}
			\begin{itemize}
				\item If code compiles successfully $\rightarrow$ reward = 0.5
				\item If compilation fails $\rightarrow$ return current reward
			\end{itemize}
		\end{enumerate}
	\end{block}
\end{frame}

\begin{frame}
	\frametitle{Example Implementation of Reward Function (cont.)}
	\begin{columns}
		\begin{column}{0.48\textwidth}
			\begin{block}{Reward Function Logic - Part 2}
				\begin{enumerate}\setcounter{enumi}{2}
					\item \textbf{Correctness check:}
					\begin{itemize}
						\item For each test case, compare PyTorch and Triton results
						\item If outputs match $\rightarrow$ reward = 1.0
						\item Otherwise $\rightarrow$ reward = 0.6
					\end{itemize}
				\end{enumerate}
			\end{block}
		\end{column}
		\begin{column}{0.48\textwidth}
			\begin{block}{Key Elements}
				\begin{itemize}
					\item Multi-stage evaluation process
					\item Each stage builds on the previous
					\item Partial rewards for partial successes
					\item Deterministic verification through test cases
					\item Numerically scaled from 0.0 to 1.0
					\item Multiple test cases ensure robustness
				\end{itemize}
			\end{block}
		\end{column}
	\end{columns}
\end{frame}

\begin{frame}
	\frametitle{Training Loop and Results}
	\begin{block}{How GRPO (Gradient-based Reward Policy Optimization) Works:}
		\begin{itemize}
			\item \textbf{Generate}: Multiple completions per prompt using sampling
			\item \textbf{Evaluate}: Run reward checks on each completion
			\item \textbf{Update}: Compute advantages and backpropagate signals
			\item \textbf{Repeat}: Model refines strategy to maximize rewards
		\end{itemize}
	\end{block}
	\begin{block}{Results:}
		\begin{itemize}
			\item 53\% accuracy on held-out examples after 5,000 steps
			\item 3x higher correctness rate than OpenAI o1 and DeepSeek-R1
			\item 4x better performance than Claude 3.7 Sonnet
		\end{itemize}
	\end{block}
\end{frame}

\begin{frame}
	\frametitle{\textbf{Performance Comparison}: RFT vs. Leading Models}
	\vspace{-0.2cm}
	\begin{columns}[T]
		\begin{column}{0.48\textwidth}
			\begin{center}
				\begin{tikzpicture}[scale=0.8, transform shape]
					\begin{axis}[
						ybar,
						axis lines=left,
						bar width=20pt,
						ylabel={\textbf{Accuracy (\%)}},
						symbolic x coords={RFT, o1, DeepSeek-R1, Claude 3.7},
						xtick=data,
						xtick style={draw=none},
						ytick={0,10,20,30,40,50,60},
						ymin=0,
						ymax=60,
						nodes near coords={\textbf{\pgfmathprintnumber\pgfplotspointmeta}},
						width=6.5cm,
						height=5cm,
						major grid style={draw=gray!30},
						grid=major,
						]
						\addplot[fill=blue!80, draw=blue!90!black] coordinates {(RFT, 53) (o1, 18) (DeepSeek-R1, 17) (Claude 3.7, 13)};
					\end{axis}
				\end{tikzpicture}
			\end{center}
		\end{column}
		\begin{column}{0.48\textwidth}
			\begin{block}{\textbf{RFT Training Progression}}
				\begin{itemize}\setlength{\itemsep}{0.4em}
					\item \textbf{Starting point}: 5\% accuracy
					\item \textbf{Early training} (1,000 steps): 22\%
					\item \textbf{Mid-training} (3,000 steps): 41\%
					\item \textbf{Final result}: \textcolor{blue}{53\% accuracy}
					\item Model developed generalizable patterns beyond the limited training examples
				\end{itemize}
			\end{block}
		\end{column}
	\end{columns}
\end{frame}

\begin{frame}[shrink=15]
	\frametitle{Performance Comparison with Leading Models}
	\begin{columns}[T]
		\begin{column}{0.48\textwidth}
			\begin{block}{Results:}
				\begin{itemize}\setlength{\itemsep}{0.3em}
					\item 53\% accuracy on held-out examples after 5,000 steps
					\item 3x higher correctness rate than OpenAI o1 and DeepSeek-R1
					\item 4x better performance than Claude 3.7 Sonnet
				\end{itemize}
			\end{block}
		\end{column}
		\begin{column}{0.48\textwidth}
			\begin{block}{Key Success Factors}
				\begin{itemize}\setlength{\itemsep}{0.3em}
					\item \textbf{奖励设计}: 多级奖励提供了清晰的学习信号
					\item \textbf{测试多样性}: 多样化的测试用例防止了过拟合
					\item \textbf{反奖励黑客}: 防止模型仅仅记忆输出
				\end{itemize}
			\end{block}
		\end{column}
	\end{columns}
\end{frame}

\section{使用Unsloth实现实用的RFT}

\begin{frame}[shrink=60]
	\frametitle{Unsloth实用RFT工作流程}
	\begin{columns}[T]
		\begin{column}{0.48\textwidth}
			\begin{block}{Unsloth是什么?}
				\begin{itemize}
					\item 用于高效LLM微调的开源库
					\item 可在有限硬件上运行(3GB+显存)
					\item 支持QLoRA、LoRA、RLHF、GRPO
					\item 比标准方法快2-4倍
				\end{itemize}
			\end{block}
		\end{column}
		\begin{column}{0.48\textwidth}
			\begin{block}{开发工作流程}
				\begin{enumerate}
					\item \textbf{设置}: 安装依赖项
					\item \textbf{模型}: 选择基础模型
					\item \textbf{数据}: 格式化训练数据
					\item \textbf{配置}: 设置参数
					\item \textbf{训练}: 运行微调
					\item \textbf{部署}: 保存并服务
				\end{enumerate}
			\end{block}
		\end{column}
	\end{columns}
\end{frame}

\begin{frame}[fragile]
	\frametitle{步骤1: 环境设置}
	\begin{block}{安装Unsloth}
		\begin{mintedbox}{bash}
# Basic installation
pip install unsloth

# For specific CUDA versions
# pip install unsloth[cu118]  # CUDA 11.8
# pip install unsloth[cu121]  # CUDA 12.1
		\end{mintedbox}
	\end{block}
\end{frame}

\begin{frame}[fragile]
	\frametitle{步骤1: 所需库}
	\begin{mintedbox}{python}
# 导入必要库
import torch
from unsloth import FastLanguageModel
from datasets import load_dataset
from transformers import TrainingArguments
	\end{mintedbox}
	\begin{block}{性能优化}
		\begin{itemize}
			\item 适用于显存有限设备的CPU卸载
			\item 用于更快训练的Flash Attention 2(在支持的GPU上)
			\item 梯度检查点以计算换内存
		\end{itemize}
	\end{block}
\end{frame}

\begin{frame}[fragile]
	\frametitle{步骤2: 选择模型与方法}
	\begin{columns}
		\begin{column}{0.48\textwidth}
			\begin{block}{加载基础模型}
			\begin{mintedbox}{python}
max_seq_length = 2048
model, tokenizer = FastLanguageModel.from_pretrained(
    model_name = "unsloth/llama-3.1-8b-bnb-4bit",
    max_seq_length = max_seq_length,
    load_in_4bit = True)
			\end{mintedbox}
			\end{block}
		\end{column}
		\begin{column}{0.48\textwidth}
			\begin{block}{选择微调方法}
				\begin{itemize}
					\item \textbf{QLoRA}: 与LoRA结合的4位量化(最少资源)
					\item \textbf{LoRA}: 低秩适应(质量/资源平衡)
					\item \textbf{GRPO}: 用于DeepSeek风格的强化学习
					\item \textbf{全量微调}: 在高端硬件上获得最佳质量
				\end{itemize}
			\end{block}
		\end{column}
	\end{columns}
\end{frame}

\begin{frame}[fragile]
	\frametitle{步骤2: 为RFT配置LoRA}
	\begin{block}{RFT特定配置}
		\begin{itemize}
			\item 针对键值注意力层以获得最高效的适应
			\item 对于强化学习等复杂任务使用秩16-32
			\item 添加dropout防止对奖励信号过拟合
			\item 启用梯度检查点以在训练期间节省内存
		\end{itemize}
	\end{block}
	\begin{block}{代码实现}
		\begin{mintedbox}{python}
model = FastLanguageModel.get_peft_model(
    model,
    r = 16,  # LoRA rank
    target_modules = ["q_proj", "k_proj", "v_proj", "o_proj"],
    alpha = 16,  # LoRA alpha
    dropout = 0.05,  # Add regularization
    use_gradient_checkpointing = True)
		\end{mintedbox}
	\end{block}
\end{frame}

\begin{frame}[fragile]
	\frametitle{步骤3: 数据集准备}
	\begin{block}{RFT的数据集格式}
		\begin{itemize}
			\item 格式化包含输入和预期输出的数据集
			\item 包含奖励信号或成功标准
			\item 确保不同场景下的多样化示例
		\end{itemize}
	\end{block}
	\begin{mintedbox}{python}
dataset = [
    {"input": "Solve: $2x + 3 = 7$", 
     "output": "To solve for x:\n$2x + 3 = 7$\n$2x = 4$\n$x = 2$"},
    {"input": "What is the capital of France?", 
     "output": "The capital of France is Paris."},
    # ...more examples
]
	\end{mintedbox}
\end{frame}

\begin{frame}[fragile]
	\frametitle{步骤3: 数据集格式化为训练}
	\begin{block}{应用适当的模板}
		\begin{itemize}
			\item 使用模型的特定聊天模板格式
			\item 确保正确应用特殊标记
			\item 为您的用例设置适当的序列长度
		\end{itemize}
	\end{block}
	\begin{mintedbox}{python}
formatted_dataset = FastLanguageModel.format_dataset(
    dataset,
    tokenizer,
    max_seq_length,
    add_special_tokens = True,
    template = "<s>[INST] {input} [/INST] {output} </s>"
)
	\end{mintedbox}
\end{frame}

\begin{frame}[fragile]
	\frametitle{步骤4: 配置训练参数}
	\begin{block}{训练参数}
	\begin{mintedbox}{python}
training_args = TrainingArguments(
    output_dir = "./results",
    num_train_epochs = 3,
    per_device_train_batch_size = 4,
    gradient_accumulation_steps = 2,
    learning_rate = 2e-4,
    weight_decay = 0.01,
    max_grad_norm = 0.3,
    logging_steps = 10,
    save_total_limit = 3,
)
	\end{mintedbox}
	\end{block}
\end{frame}

\begin{frame}[fragile]
	\frametitle{步骤4: RFT的超参数选择}
	\begin{columns}
		\begin{column}{0.48\textwidth}
			\begin{block}{关键超参数}
				\begin{itemize}
					\item \textbf{LoRA秩 (r)}: 在8-32之间;越高提供更强的表达能力
					\item \textbf{LoRA Alpha}: 通常与秩相同
					\item \textbf{学习率}: QLoRA为2e-4至5e-4
					\item \textbf{训练轮数}: 小数据集为2-5轮
					\item \textbf{Dropout}: 0.05-0.1用于正则化
				\end{itemize}
			\end{block}
		\end{column}
		\begin{column}{0.48\textwidth}
			\begin{block}{Avoid Overfitting in RFT}
				\begin{itemize}
					\item Increase dropout if showing signs of overfitting
					\item Implement early stopping by monitoring loss
					\item Use a validation set to evaluate generalization
					\item Add weight decay for regularization
				\end{itemize}
			\end{block}
		\end{column}
	\end{columns}
\end{frame}

\begin{frame}[fragile]
	\frametitle{Step 5: Training Execution}
	\begin{block}{Execute Training}
		\begin{mintedbox}{python}
from trl import SFTTrainer

trainer = SFTTrainer(
    model = model,
    train_dataset = formatted_dataset,
    args = training_args,
    tokenizer = tokenizer,
    packing = False,
    dataset_text_field = "text",
)
		\end{mintedbox}
	\end{block}
\end{frame}

\begin{frame}[fragile]
	\frametitle{Step 5: Training Process and Monitoring}
	\begin{block}{Run Training}
		\begin{mintedbox}{python}
# Start training
trainer.train()
		\end{mintedbox}
	\end{block}
	\begin{block}{Monitor Key Metrics}
		\begin{itemize}
			\item \textbf{Training Loss}: Should steadily decrease but not plateau too quickly
			\item \textbf{Validation Loss}: Monitor for signs of overfitting (uptick in validation loss)
			\item \textbf{Learning Rate}: Record effect of LR variations on performance
			\item \textbf{GPU Utilization}: Verify efficient resource usage
		\end{itemize}
	\end{block}
\end{frame}

\begin{frame}[fragile]
	\frametitle{Step 6: Evaluation}
	\begin{block}{Evaluate Model Performance}
		\begin{itemize}
			\item Load the fine-tuned model for inference
			\item Create structured test prompts
			\item Compare with baseline model on same inputs
			\item Test diverse scenarios and edge cases
			\item Measure inference speed and resource usage
		\end{itemize}
	\end{block}
	\begin{mintedbox}{python}
# Example test inference
prompt = "Solve: 3x + 5 = 20"
response = model.generate_text(prompt)
# Result: "To solve: 3x + 5 = 20
Subtract 5: 3x = 15
Divide by 3: x = 5"
	\end{mintedbox}
\end{frame}

\begin{frame}[fragile]
	\frametitle{Step 6: Deployment}
	\begin{block}{Save and Deploy Model}
		\begin{itemize}
			\item Save trained model and tokenizer
			\item Optimize for inference (e.g., ONNX conversion) 
			\item Configure deployment environment
			\item Set up monitoring and logging
		\end{itemize}
	\end{block}
	\begin{mintedbox}{python}
# Save the fine-tuned model
model.save_pretrained("./my-rft-model")
tokenizer.save_pretrained("./my-rft-model")
	\end{mintedbox}
\end{frame}

\begin{frame}
	\frametitle{Step 6: Continuous Improvement}
	\begin{block}{Ongoing Refinement}
		\begin{itemize}
			\item Collect user feedback and model outputs in production
			\item Update reward functions based on real-world performance
			\item Expand training dataset with edge cases discovered in deployment
			\item Periodically retrain to incorporate improvements
		\end{itemize}
	\end{block}
	\begin{block}{Long-term Maintenance}
		\begin{itemize}
			\item Monitor for performance degradation over time
			\item Keep track of new best practices in RFT 
			\item Evaluate benefits of updating to newer base models
		\end{itemize}
	\end{block}
\end{frame}

\begin{frame}
	\frametitle{Ready-to-Use Unsloth Notebooks}
	\begin{block}{Available Implementation Resources}
		\begin{itemize}
			\item Unsloth provides ready-to-use notebooks for various models and tasks
			\item Accessible via Google Colab or Kaggle (free GPU resources)
			\item Includes both fine-tuning and GRPO (RFT) implementations
		\end{itemize}
	\end{block}
	\begin{columns}
		\begin{column}{0.48\textwidth}
			\begin{block}{Popular Models}
				\begin{itemize}
					\item Llama 3.1 (8B)
					\item Phi-4 (14B)
					\item Mistral (7B, 22B)
					\item Qwen 2.5 (3B, 14B)
					\item Gemma 2 (2B, 9B)
				\end{itemize}
			\end{block}
		\end{column}
		\begin{column}{0.48\textwidth}
			\begin{block}{Specialized Variants}
				\begin{itemize}
					\item Qwen2.5-Coder (14B)
					\item CodeGemma (7B)
					\item Llama 3.2 Vision
					\item Qwen2-VL (7B)
					\item Phi-3 Medium
				\end{itemize}
			\end{block}
		\end{column}
	\end{columns}
	\begin{center}
		\small{Notebooks available at: \texttt{https://docs.unsloth.ai/get-started/unsloth-notebooks}}
	\end{center}
\end{frame}

\begin{frame}
	\frametitle{Dataset Construction for RFT}
	\begin{block}{Getting Started}
		\begin{itemize}
			\item Identify the purpose of your dataset: chat dialogues, structured tasks, or domain-specific data
			\item Define desired output style: JSON, HTML, text, code or specific languages
			\item Determine data sources: Hugging Face datasets, Wikipedia, or synthetic data
		\end{itemize}
	\end{block}
	\begin{columns}
		\begin{column}{0.48\textwidth}
			\begin{block}{Common Data Formats}
				\begin{itemize}
					\item Text-only format
					\item Instruction-Input-Output
					\item ShareGPT format (multi-turn)
					\item ChatML (OpenAI style)
				\end{itemize}
			\end{block}
		\end{column}
		\begin{column}{0.48\textwidth}
			\begin{block}{Dataset Requirements}
				\begin{itemize}
					\item Minimum: 100 examples
					\item Optimal: 1,000+ examples
					\item Quality over quantity
					\item Can combine multiple datasets
				\end{itemize}
			\end{block}
		\end{column}
	\end{columns}
\end{frame}

\begin{frame}
	\frametitle{RFT Implementation: Common Pitfalls (1)}
	\begin{itemize}
		\item \textbf{Problem}: Model produces generic or refuses to generate responses\\
		  \textbf{Solution}: Increase LoRA rank and alpha; ensure diverse training data
		\item \textbf{Problem}: Catastrophic forgetting (model loses pre-trained capabilities)\\
		  \textbf{Solution}: Lower learning rate; add regularization
		\item \textbf{Problem}: High training loss that doesn't converge\\
		  \textbf{Solution}: Check dataset formatting; reduce sequence length
	\end{itemize}
\end{frame}

\begin{frame}[shrink=10]
	\frametitle{RFT实施:常见问题 (2)}
	\begin{itemize}
		\item \textbf{问题}: 训练期间内存不足错误\\
		  \textbf{解决方案}: 启用梯度检查点;减小批量大小
		\item \textbf{问题}: 模型生成幻觉\\
		  \textbf{解决方案}: 实施惩罚虚构内容的奖励函数
		\item \textbf{问题}: 使用奖励信号时训练不稳定\\
		  \textbf{解决方案}: 归一化奖励;使用带有适当裁剪的PPO
	\end{itemize}
\end{frame}

\section{结论}

\begin{frame}[shrink=60]
	\frametitle{关键要点}
	\vspace{-0.2cm}
	\begin{columns}[T]
		\begin{column}{0.48\textwidth}
			\begin{block}{RFT的力量:}
				\begin{itemize}\setlength{\itemsep}{-0.1em}
					\item \textbf{自我提升}: 模型超越静态方法
					\item \textbf{数据效率}: 使用更少数据超越SFT
					\item \textbf{速度}: Turbo LoRA将处理量提高2-4倍
					\item \textbf{实际应用}: 超越学术领域
				\end{itemize}
			\end{block}
		\end{column}
		\begin{column}{0.48\textwidth}
			\begin{block}{范式转变:}
				\begin{itemize}\setlength{\itemsep}{-0.1em}
					\item \textbf{传统方式}: 大数据 $\rightarrow$ 静态模型
					\item \textbf{RFT}: 最小数据 + 奖励 $\rightarrow$ 成长
					\item \textbf{新循环}: 持续改进
				\end{itemize}
			\end{block}
		\end{column}
	\end{columns}
\end{frame}

\begin{frame}[shrink=10]
	\frametitle{实施策略 - 核心组件}
	\begin{block}{所需核心组件}
		\begin{enumerate}
			\item 基础模型(开源LLM)
			\item 奖励函数定义
			\item 提示数据集(可以很小)
			\item 强化学习算法(RLHF、GRPO等)
			\item 评估框架
		\end{enumerate}
	\end{block}
\end{frame}

\begin{frame}[shrink=10]
	\frametitle{实施策略 - 最佳实践}
	\begin{block}{最佳实践}
		\begin{itemize}
			\item 从小型、清晰的奖励函数开始
			\item 构建全面的验证测试
			\item 实施反奖励黑客措施
			\item 监控演发行为
			\item 逐步增加复杂性
		\end{itemize}
	\end{block}
\end{frame}

\begin{frame}[shrink=10]
	\frametitle{RFT的后续步骤}
	\begin{block}{开始使用:}
		\begin{itemize}
			\item 强化微调标志着LLM发展的重大飞跃
			\item 通过奖励信号而非标记示例进行训练
			\item 端到端平台使开发者能够轻松使用这种方法
			\item 专注于创新而非基础设施复杂性
		\end{itemize}
	\end{block}
	\begin{block}{RFT应用的成熟领域}
		\begin{itemize}
			\item \textbf{专业编程}: 领域特定的代码生成(嵌入式系统、高性能计算)
			\item \textbf{科学研究}: 提出并验证假设的模型
			\item \textbf{推理任务}: 复杂的逻辑和数学问题解决
			\item \textbf{教育}: 能够理解学生知识空白的自适应辅导系统
		\end{itemize}
	\end{block}
\end{frame}

\begin{frame}{}
	\begin{center}
		\LARGE Thank You!
	\end{center}
	\begin{center}
		\large Questions?
	\end{center}
\end{frame}

\end{document}
